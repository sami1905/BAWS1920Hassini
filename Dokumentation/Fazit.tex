\section{Fazit}
	Das Fazit sollte den Grad der Zielerreichung widerspiegeln. Hierzu wird auf die in der Einleitung und in der Zielhierarchie gesetzten Ziele verwiesen und die abgeschlossene Arbeit bewertet.\\
	Zu Beginn des Projekts wurde das primäre Ziel festgelegt, den Anwender bei der Behandlung von Diabetes mit dem zu entwickelnden System zu unterstützen. Es wurde ein System entwickelt und implementiert, um Ereignisse wie Blutzuckerspiegel, Mahlzeiten und sportliche Aktivitäten zu dokumentieren. Zusätzlich zur Dokumentation bietet das System kurzfristig die Berechnung des Kalorienbedarfs und eine langfristige Unterstützung für den Lebensstil des Benutzers. Zu diesem Zweck wird der Body-Mass-Index des Benutzers anhand seines Gewichts berechnet und ein Gewichtsziel vorgeschlagen. Basierend auf dem Gewichtsziel wird der Kalorienbedarf so angepasst, dass das System den optimalen Kohlenhydrat-, Protein- und Fettbedarf für den Benutzer berechnet, wenn das Körpergewicht erhöht oder verringert wird.\\
	Ein weiteres Alleinstellungsmerkmal ist die Kontaktaufnahme mit anderen Diabetikern. Hier wurde ein „Wartezimmer“ implementiert, in dem Benutzer Beiträge, Fragen und Erfahrungen posten und kommentieren können. Mit diesen beiden Alleinstellungsmerkmalen bietet das System einen beispiellosen Umfang für die Behandlung von Diabetes mit einem technischen Hilfsmittel.\\
	Schaut man sich die operativen Ziele der Zielhierarchie an, sind alle Ziele erreicht. Sowohl der BE/KE- und Insulineinheiten- als auch der HbA1c-Rechner wurden erfolgreich implementiert. Neben Informationen zu Mahlzeiten und sportlichen Aktivitäten werden dem Benutzer Daten zur Insulin- und BE/KE-Einnahme angezeigt.\\
	Durch die Möglichkeit benutzerbezogene Daten eingeben zu können, bietet das System jedem Benutzer Individualität und durch Erfahrungsaustausch können sich Diabetiker gegenseitig beraten, so dass auch Teile der taktischen Ziele erreicht werden konnten.\\
	Mit definierten Standards und Konventionen im Design bietet das System dem Benutzer eine intuitive Benutzeroberfläche, die eine transparente und zeitsparende Dokumentation ermöglicht.\\
	Allerdings konnten keine externen API-Schnittstellen verwendet werden, da weder eine Lebensmittel- noch eine Aktivitätsdatenbank vorhanden ist. Es war auch nicht möglich, Blutzuckerwerte von externen Blutzuckermessgeräten zu importieren. Hier sollte die Dexcom-Schnittstelle verwendet werden. Anders als in der ersten Entwicklungsphase ist dies jedoch in Europa aufgrund der Datenschutzverordnung nicht möglich. Trotzdem wurde ein System implementiert, das gesunde Ernährung und Sport ermöglicht und den Austausch zwischen Diabetikern verbessert.\\
	Um messen zu können, ob sich das System positiv auf den HbA1c-Wert und die Blutzuckerwerte der Anwender auswirkt und eine Aufrechterhaltung der Lebensqualität gewährleisten kann, müsste eine Usability-Evaluation durchgeführt werden, indem das entwickelte System von Probanden getestet und bewertet wird. \\
	Im Idealfall sollten benutzerbezogene Daten wie E-Mail-Adresse und Passwort verschlüsselt werden, bevor das System zur Verfügung gestellt wird. Verschlüsselung war nicht Teil der Projektanforderungen. Es wird empfohlen, dies in neuem System-Release zu implementieren.\\
	Zusammenfassend lässt sich sagen, dass viele Ideen und Überlegungen erfolgreich umgesetzt wurden, vor allem aber nach wissenschaftlichen Richtlinien. Der Bereich Medizin und Gesundheit bietet einen derart umfassenden Rahmen, dass dieses Projekt in noch größerem Umfang umgesetzt werden kann.
