
	\section{Methodischer Rahmen}
	\label{section:Rahmen}
	Der methodische Rahmen dient zur Festlegung der zu verwendenden Modelle aus der Mensch-Computer-Interaktion und enthält die Begründung dieser Festlegung. Zunächst wird der Nutzungskontext des zu entwickelnden Systems betrachte und sich folglich anhand des Nutzungskontext für ein Vorgehensmodell entschieden. Abschließend werden für einzelne Aktivitäten des Vorgehensmodells, für das sich entschieden wurde, angemessene Methoden gewählt und diese Wahl begründet.
	\subsection{Nutzungskontex}
	Das zu entwickelnde System soll zur Verwaltung der Blutzuckerwerte eines Diabetikers dienen. Anhand der Domänenrecherche und dessen Stakeholder ist die Zielgruppe des Systems klar erkenntlich. Benutzer des Systems sind sämtliche Diabetiker und Einzelpersonen in dessen Umfeld. Das System soll für eine Steigerung der Lebensqualität des Benutzers und auch Einzelpersonen in dessen Umfeld unter Einfluss der Diabetes des Benutzers sorgen. Um dies zu erreichen, muss das System folgende Kriterien aufweisen:
	\begin{itemize}
		\item Der Hauptfokus des Systems liegt bei dem Benutzer, dessen Aufgaben, Ziele und Eigenschaften.
		\item Das System sollte gebrauchstauglich und zweckdienlich für den Benutzer sein und für eine Verbesserung dessen User Experience sorgen.
		\item Konzentration liegt auf dem Benutzer, dessen Erfordernisse und Anforderungen.
		\item Verbesserung des menschlichen Wohlbefindens und eine Zufriedenstellung des Benutzers bewirken.
		\item Positive Auswirkung auf die Gesundheit des Benutzers.
	\end{itemize}
	\subsection{Ansätze des usage-centered-design}
	Das Usage-centered-design ist ein Vorgehensmodell, welches den Fokus auf die Funktionalität und den Verwendungszweck eines Systems legt und bei dem die Durchführung der Aufgaben des Benutzers im Zentrum stellt. Da aus der Beschreibung des Nutzungskontextes zu entnehmen ist, dass das zu entwickelnde System den Schwerpunkt auf den Benutzer, dessen Aufgaben, Ziele und Eigenschaften legt, ist die Verwendung des Usage-centered-design ausgeschlossen. \emph{[Prof. Dr. Hartmann, Gerhard: Vorlesungsbegleitende Materialien zum Modul Mensch-Computer Interaktion, 2016.]}
	\subsection{Ansätze des user-centered-design}
	Bei dem user-centered-design stimmen die Kriterien des Nutzungskontextes und die Ansätze der Modellart überein, wodurch bei Verwendung eines benutzer-zentriertem Vorgehensmodell eine erfolgreiche Umsetzung des Systems wahrscheinlicher. Folglich werden verschiedene user-centered-designs abgewogen und sich abschließend für die Verwendung eines dieser Modelle entschieden. \emph{[Prof. Dr. Hartmann, Gerhard: Vorlesungsbegleitende Materialien zum Modul Mensch-Computer Interaktion, 2016.]}
	\subsubsection{Usability-engineering nach Rosson und Carrol}
	Das usabulity-engineering-Modell nach Rosson und Carrol ist ein scenario-basiertes Modell und dient zum Verstehen, Beschreiben und Modellieren des menschlichen Handels. Es stellt die Gebrauchstauglichkeit eines Systems in den Vordergrund und eignet sich somit für die Entwicklung dieses Projektes. Allerdings ist das scenario-basierte Vorgehensmodell mit user-centered-design-Ansätzen aufwendig und bringt eine gewisse Tiefe in den einzelnen Schritten, da jede Prozessstufe von scenario-basierten Aktivitäten abhängt, mit sich. Da die zeitliche Kapazität dieses Projektes begrenzt und das zu entwickelnde System in einem kurzen Zeitraum zu entwickeln ist, eignen sich dieses Vorgehensmodell im voller Verwendung nicht für dieses Projekt. Eine Verwendung des Modells in einzelnen Entwicklungsaktivitäten ist jedoch möglich und eine Teilverwendung des usability-engineering-Modells sollte in den Entwicklungsphasen abgewägt werden. \emph{[Prof. Dr. Hartmann, Gerhard: Vorlesungsbegleitende Materialien zum Modul Mensch-Computer Interaktion, 2016.]}
	\subsubsection{Discount usability-engineering nach Nielsen}
	Auch das discount usability-engineering-Modell von Nielsen zählt zu den Vorgehensmodellen des user-centered-design und legt wie auch Rosson und Carrol die Gebrauchstauglichkeit eines Systems in den Vordergrund. Dabei wird schon früh im Entwicklungsprozess der Benutzer in den Mittelpunkt gesetzt. Da, durch den wenigen Aufwand und der fehlenden Tiefe der einzelnen Prozessstufen des Modells, die Gefahr besteht, nicht die maximale Gebrauchstauglichkeit eines Systems zu erhalten, fällt dieses Modell für das zu entwickelnde System weg. Denn die höchst mögliche Gebrauchstauglichkeit ist notwendig, um die Verwendung des Systems für den Benutzer so einfach wie möglich gestalten zu können. \emph{[Prof. Dr. Hartmann, Gerhard: Vorlesungsbegleitende Materialien zum Modul Mensch-Computer Interaktion, 2016.]}
	\subsubsection{Usability engineering lifecycle nach Mayhew}
	Das usability engineering lifecycle von Mayhews beruht vollständig auf die User-centered-design-Prinzipien- und -methoden und stellt die Gebrauchstauglichkeit eines Systems anhand objekt-orientierter Entwicklungsprozesse her. Das Modell ist iterativ und durch die drei fundamentalen Prozess-Bestanteile wird eine gut strukturierte Entwicklung ermöglicht. Da der Zeitrahmen dieses Projektes begrenzt ist, stellt die Skalierbarkeit des Modells neben der benutzer-zentrierten Methoden die Gründe für die Verwendung dieses Vorgehensmodells in diesem Projekt. Trotz der geringen Zeitkapazität weist Mayhews Modell einen hohen Detailgrad in den einzelnen Entwicklungsphasen auf und sorgen auch bei einer Anpassung des Modells eine hohe Kontrolle über die Entwicklungsphase. \emph{[Prof. Dr. Hartmann, Gerhard: Vorlesungsbegleitende Materialien zum Modul Mensch-Computer Interaktion, 2016.]}\\
	Abschließend ist zu sagen, dass das Modell nach Mayhews als Hauptvorgehensmodell des Projektes dienen und durch die Ergänzung der scenario-basierten-Ansätzen nach Rosson und Carroll eine hohe Wahrscheinlichkeit auf Gebrauchstauglichkeit des Systems ermöglicht. Mit diesem Modell werden eine erfolgreiche Umsetzung des Projektes und eine Verringerung des Risikos, dass das System von Benutzern nicht verwendet wird erhofft.