\section{Zielhierarchie}
\label{section:Zielhierarchie}
In der Zielhierarchie werden die Ziele dieses Projektes anhand ihrer Fristigkeiten gegliedert. Das Leitbild dieser Arbeit ergibt sich aus den strategischen Zielen (langfristig), den taktischen Zielen (mittelfristig) und den operativen Zielen (kurzfristig). Die Zielsetzung ist essential zu abschließenden Ermittlung des Grades der Zielerfüllen. Die Zielhierarchie bewirkt eine Abhängigkeit zwischen einzelnen Zielen aus unterschiedlichen Hierarchieebenen.
\subsection{Strategische Ziele}
\begin{itemize}
	\item \lbrack \textbf{S1}\rbrack  \textbf{ Umgang mit Diabetes mellitus erleichtern}\\
	Im Umgang mit dem Diabetes gibt es einige Aspekte zu beachten. Neben dem Kontrollieren, Dokumentieren und Verbessern der Blutzuckerwerte, spielen eine ausgewogene Ernährung, reichlich Bewegung und die eigene Zufriedenheit eine große Rolle. Eiemn Diabetiker soll damit der Umgang mit seiner Krankheit erleichtert werden.
	\item \lbrack \textbf{S2}\rbrack  \textbf{ Erhalt der Lebensqualität}\\
	Im Leben eines Diabetikers kann es schnell zu Komplikationen und Unzufriedenheit kommen. Oft führen schlechte Therapien und fehlende Disziplin zur Unzufriedenheit und zu einer niedrigen Lebensqualität. Der Erhalt der Lebensqualität einer der wichtigsten Aspekte einer erfolgreichen Therapie. Im Idealfall kann diese sogar gesteigert werden.
	\item \lbrack \textbf{S3}\rbrack  \textbf{ Anpassungen in der Lebensführung} \\
	Für eine erfolgreiche Therapie des Diabetes mellitus wird oft auf Dinge verzichtet, die Auswirkungen auf Therapie, Moral und Leben eines Diabetikers haben. Durch einfache Änderungen im Leben lässt sich dies jedoch vermeiden.  Durch die Darstellung der notwendigen Eigenschaften im Umgang mit Diabetes, sollen dem Erkrankten die Entscheidungen in der Behandlung und im Leben erleichtern werden.
\end{itemize}
\subsection{Taktische Ziele}
\begin{itemize}
	\item \lbrack \textbf{T1}\rbrack  \textbf{ Positive Auswirkung auf Blutzuckerwerte} \\
	Ein gesunder Blutzuckerwert liegt zwischen 80 und 120 mg/dl bzw. zwischen 4,4 und 6,7 mmol/l. Die Anzahl der Blutzuckerwerte im optimalen Bereich soll bei mindestens 70\% liegen. Die Anzahl der Blutzuckerwerte im grenzwertigen Bereich von 60-180mg/dl bzw. 3,3-10 mmol/l soll bei mindestens 80\% liegen.\newline
	\emph{In Abhängigkeit von: S1, S2} 
	\item \lbrack \textbf{T2}\rbrack  \textbf{ Positive Auswirkung auf HbA1c-Wert} \\
	Ein gesunder HbA1c-Wert liegt bei etwa 5\%. Bei einem Diabetiker sollte der HbA1c-Wert zwischen 6,5\% und 7,5\% liegen. \newline
	\emph{In Abhängigkeit von: S1, S2} 
	\item \lbrack \textbf{T3}\rbrack  \textbf{ Einfache, transparente und zeitgewinnende Dokumentation} \\
	Ein Diabetiker sollte nicht länger als 20 Minuten pro Tag mit der Dokumentation seiner Therapie verbringen.\newline
	\emph{In Abhängigkeit von: S1, S2, S3} 
	\item \lbrack \textbf{T4}\rbrack  \textbf{ Beratung} \\
	Bei der Behandlung von Diabetes mellitus müssen viele Entscheidungen getroffen werden. Jede Entscheidung hat Folgen für die Therapie des Diabetes mit sich. Durch Beratung bzw. Einbringung von Dritten, die ebenfalls betroffen sind und bereits eigene Erfahrungen im Umgang mit dem Diabetes haben sammeln können, sollen Entscheidungen leichter getroffen werden. \newline
	\emph{In Abhängigkeit von: S1, S2, S3} 
	\item \lbrack \textbf{T5}\rbrack  \textbf{ Individualität} \\  
	Jeder menschliche Körper unterscheidet sich und reagiert auf Außeneinflüsse anders. So hat beispielsweise Insulin bei jedem Körper eine andere Wirkung. Durch individuelle Benutzerdaten, wie BE- und Insulinfaktoren, sollen Berechnungen und Prozesse in der Therapie speziell auf jeden Diabetiker abgestimmt werden. \newline
	\emph{In Abhängigkeit von: S1, S2, S3} 
\end{itemize}

\subsection{Operative Ziele}
\begin{itemize}
	\item \lbrack \textbf{O1}\rbrack  \textbf{ BE-Rechner} \\
	BE’s, auch Broteinheiten genannt, lassen sich aus der Grammanzahl an Kohlenhydrate berechnen. In der Ernährung kommt es oftmals zu falschen Berechnungen durch den Patienten. Ein BE-Rechner dient in erster Linie dazu, dem Diabetiker die Berechnungen abzunehmen und garantiert eine korrekte Berechnung.\newline
	\emph{In Abhängigkeit von: T1, T2, T3, T5} 
	\item \lbrack \textbf{O2}\rbrack  \textbf{ Insulineinheiten-Rechner} \\
	Insulin sorgt für den Abbau und Transport des Zuckers im Blut, indem es ihn aus der Blutbahn entnimmt und zur Energieversorgung zu den Muskelzellen transportiert. Jeder Körper reagiert unterschiedlich auf eine gewisse Menge an Insulin. Man spricht vom Insulinfaktor jedes Menschen. Dieser Faktor muss mit der BE-Anzahl multipliziert werden, um den Wert des Anstiegs des Blutzuckers pro BE zu bestimmen. Auch hier werden oft Fehlberechnungen durchgeführt, wodurch Blutzuckerwerte außerhalb des Zielbereiches entstehen. Durch einen Insulineinheiten-Rechner unter Angabe der individuellen Insulinfaktoren, soll eine korrekte Berechnung der erforderlichen Insulineinheiten gewährleistet sein.\newline
	\emph{In Abhängigkeit von: T1, T2, T3, T5} 
	\item \lbrack \textbf{O3}\rbrack  \textbf{ HbA1c-Rechner} \\
	Der HbA1c-Wert gibt an, wie viel Zucker sich an den roten Blutkörperchen im menschlichen Körper angesetzt haben. Durch den HbA1c-Wert erhält man einen durchschnittlichen Blutzuckerwert der letzten 6-9 Wochen. Mithilfe  des berechneten HbA1c-Wertes aus den eingelesenen Blutzuckerwerten  der letzten Wochen, kann dem Patienten die Qualität seiner Blutzuckerwerte präsentiert werden.\newline
	\emph{In Abhängigkeit von: T1, T2, T3, T5} 
	\item \lbrack \textbf{O4}\rbrack  \textbf{ Übersicht der Insulin- und BE-Einnahmen} \\
	Die Mengen der zu sich genommenen Insulineinheiten und der BE’s sind Indikatoren für die Qualität der Ernährung. Tägliche Vergleiche geben einen Überblick über die Kohlenhydrataufnahme und die Veränderung der Insulinresistenz. In Verbindung mit der Dokumentation der Blutzuckerwerte kann die Übersicht über die täglichen Insulin-und BE-Einnahmen, Eingriffe in die Therapie begründen.\newline
	\emph{In Abhängigkeit von: T1, T3, T5} 
	\item \lbrack \textbf{O5}\rbrack  \textbf{ Zugriff auf Lebensmitteldatenbank} \\
	Eine Lebensmitteldatenbank ermöglicht jederzeit die Berechnung der BE‘s einer Mahlzeit. Eine  Stammdatenbank  mit Nährstoffwerten sorgt für korrekte Berechnung der erforderlichen BE’s und Insulineinheiten ohne  interaktive Eingabe durch den Patienten. Diese Datenbank kann individuell erweitert werden und ermöglicht die Aufrechterhaltung von Essgewohnheiten und verbessert die Lebensqualität durch die Pflege von Lebensmitteln, die häufig vom Patienten konsumiert werden.\newline
	\emph{In Abhängigkeit von: T1, T2, T3, T5} 
	\item \lbrack \textbf{O6}\rbrack  \textbf{ Kommunikation/Therapieempfehlungen} \\
	Die Kommunikation unter Diabetikern ist wichtig. Viele Prozesse in der Behandlung können durch persönliche Erfahrungen verändert und verbessert werden. Theoretische Ansätze von Ärzten werden oft erst in der Praxis optimal angepasst. So sind Erfahrungsaustausch und Therapieempfehlung für einen Diabetiker von einem Diabetiker oft von großer Bedeutung. \newline
	\emph{In Abhängigkeit von: T4} 
	\item \lbrack \textbf{TO7}\rbrack  \textbf{ Sport- und Ernährungsdokumentation} \\
	Bewegung und eine ausgewogene Ernährung sind sehr wichtig für einen gesunden Lebensstil. Mangelnde Aktivität und eine schlechte Ernährung führen bei einem Diabetiker zu einer schnellen Gewichtszunahme und eine erhöhte Insulinresistenz. Der Sport und die Ernährung haben somit einen großen Einfluss auf das Leben eines Diabetikers. Gerade bei der Ernährung fällt es vielen Menschen schwer auf etwas zu verzichten. Die Dokumentation von sportlichen Aktivitäten und der Ernährung soll das Risiko von Übergewicht und Folgeerkrankungen verringern und die Erhaltung der Lebensqualität sicherstellen.\newline
	\emph{In Abhängigkeit von: T1, T2, T3, T5} 
\end{itemize}