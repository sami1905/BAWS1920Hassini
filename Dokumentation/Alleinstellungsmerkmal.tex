\section{Alleinstellungsmerkmal}
	Betrachtet man die Marktrecherche und die darin vorgestellten Systeme, so erkennt man, dass einzelne Funktionen in den Anwendungen teilweise  zur Lösung des Nutzungsproblems beitragen, aber keine einzelne Anwendung hat im aktuellen Anwendungsbereich den Umfang einer optimalen Lösung. Das Kombinieren der verschiedenen Unterfunktionen mehrerer verschiedener Systeme wäre hier ein erstes Alleinstellungsmerkmal.\newline
	In Anbetracht der derzeitigen Systeme auf dem Markt für die Behandlung von Diabetes würden einige Alleinstellungsmerkmale hinzugefügt, die sich wie folgt beschreiben lassen:
	\begin{itemize}
	\item Eine Lebensmitteldatenbank, die Lebensmittel und ihre Nährwerte wie Kalorien und Kohlenhydrate enthält, um zu dokumentieren, was konsumiert wurde, und um Informationen für einen Kalorienzähler bereitzustellen. Der Benutzer sollte in der Lage sein, diese Lebensmitteldatenbank individuell zu erweitern. 
	\item Eine Aktivitätsdatenbank mit einer Auswahl an Sportaktivitäten berechnet einen Kaloriensatz der während einer bestimmten Dauer der Aktivität verbrauchten Kalorien. Diese verbrannten Kalorien sollten auf dem Kalorienzählerkonto ausgewiesen werden.
	\item Ein Kalorienzähler zur Darstellung des Grundbedarfs an Kalorien. Dieser wird  nach Aufnahmen von Mahlzeiten reduziert und durch erfolgten Leistungsumsatz erhöht werden, um  dem Benutzer einen Überblick über die verfügbaren Kalorien zu geben.
	\item Ein Zähler für Kohlenhydrate, Eiweiß und Fett basierend auf empfohlenen Werten. Er nimmt mit der Aufnahme von Nährwerten über die Nahrung ab. 
	\item Eine Plattform, um Erfahrungen auszutauschen, Kontakte zu knüpfen und mit Gleichgesinnten zu kommunizieren. 
	\item Alle Funktionen sind in einem einzigen System verfügbar.
	\end{itemize}