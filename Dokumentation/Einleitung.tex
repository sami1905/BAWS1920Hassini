\section*{Einleitung}\markboth{Einleitung}{Einleitung}\addcontentsline{toc}{section}{Einleitung}
%Die Einleitung fungiert als EInführung in das Thema, Rechtfertigung der Themenstellung sowie der Forschungsfrage und soll den Bezug zur aktuellen Diskussion herstellen. DIe Einleitung umfasst drei Aspekte: \\
%Relevanz: Warum ist das Thema überhaupt wichtig?\\
%orschungsfrage: Welche Fragen will die Arbeit beantworten?\\
%Vorgangsweise: Wie gehe ich beim Bearbeiten und Beantworten der Fragen vor?\\
%\\
%\\
\subsection{Problemstellung}
	Diabetes mellitus stammt aus dem Altgriechischen und bedeutet wörtlich übersetzt „honigsüßer Durchfluss“. Gemeint ist damit, viel süß schmeckender Urin. Bekannt ist die Stoffwechselerkrankung schon seit dem Altertum, jedoch waren jegliche Ursachen unbekannt und die Behandlung unmöglich. \newline 
	Der Diabetes ist eine Störung des Stoffwechsels, wodurch kein eigenes Insulin mehr vom Körper produziert werden kann oder der menschliche Organismus gegen dieses eine Resistenz bildet. Insulin ist ein Hormon und dient zur Regulierung des Blutzuckers. Es transportiert den Zucker aus dem Blut in die Zellen der Muskelatur und versorgt diese mit der notwendigen Energie. Der Anteil des Zuckers in Blut ist gesund, wenn er überwiegend zwischen 60 mg/dL und  140 mg/dL (3,4-7,8 mmol/L) liegt. Der Zucker gelangt nach der Essensaufnahme und nach dem Abbau der Kohlenhydrate in Glukose folglich in Blut und Leber. Wird die Zuckerzufuhr der Muskelatur gestört, beginnt das Blut zu übersäuern und es entsteht das Risiko einer Ketoacidose. Folglich gelangen Ketone in die Blutbahn und in den Urin, welche Organe und Körperbestandteile beschädigen. Bei Nichtbehandlung ist Koma oder sogar der Tod die Folge. Insulin wurde einem Menschen erstmals 1922 erfolgreich gespritzt. Erst 1996 kamen sogenannte Kunstinsuline auf den Markt, welche bis heute die Grundlage der Behandlgung von Diabetes mellitus darstellen. \newline
	Noch heute ist Diabetes eine rätselhafte Erkrankung, da nach wie vor nicht alle Fragen beantwortet sind und vorallem die Entstehung der Folgeerkrankungen ungeklärt blieb. \cite{SG}\\
	Die Prävalenz von Diabetes nimmt stetig zu. Laut der International Diabetes Federation (im Nachfolgenden IDF) beläuft sich die Zahl der an Diabetes mellitus erkrankten zwischen 20 und 79 Jahren im Jahre 2017 weltweit auf knapp 425 Millionen 
	und somit 8,8\% der Gesamtbevölkerung. Vergleicht man die Prävalenz aus dem Jahre 2017 mit der aus 1980, hat sich die Zahl der Diabetiker fast vervierfacht.\cite[S. 9]{IDF}\newline
	In Europa sind 2017 rund 6,8\% und somit 58 Millionen der 20 bis 79-Jährigen an Diabetes erkrankt und die IDF prophezeit 2045 etwa 67 Millionen Erkrankungen, ein Anstieg von 16\%. In Deutschland leben sogar rund 8,3\% der Bevölkerung zwischen 20 und 79 Jahren mit der Stoffwechselkrankheit. Dabei schätzt die IDF, dass im Jahre 2017 weitere 212,4 Millionen Erkrankungen weltweit und 22 Millionen in Europa noch nicht diagnostiziert sind.\cite[S. 110 ff.]{IDF}\newline
	Die Zahl der Todesfälle durch Diabetes mellitus zeigt, dass er als Todesursache unterschätzt wird. Laut der IDF belaufen sich die Todesfälle aufgrund von Diabetes weltweit auf rund 4 Millionen, in Europa auf rund 477,7 Tausend und in Deutschland auf 40,2 Tausend Menschen.\cite[S. 46]{IDF}\\
	Diese Zahlen zeigen, dass diese Stoffwechselstörung trotz des wissenschaftlichen und technologischen Fortschritts in der Medizin zahlreiche Komplikationen mit sich bringt und das Leben der Betroffenen extrem beeinträchtigt. Neben dem Risiko schwerwiegender Folgeerkrankungen gehören überwältigende Alltagssituationen, Ernährungsschwierigkeiten und das Verhindern oder Unterlassen von Sport zu den Ursachen für die sinkende Lebensqualität der Kranken. \newline
	Während der Diabetiker aufgrund von Überzuckerungen (Hyperglykämie) bzw. Unterzuckerungen (Hypoglykämie) jedesmal gezwungen wird, sportliche Aktivitäten einzustellen, verursachen falsch berechnete Insulineinheiten während der Therapie schlechte Blutzuckerwerte. Oft mangelt es an Möglichkeiten, Erfahrungen unter Diabetikern auszutauschen und Fragen zu neuen Behandlungsmethoden oder neuen Technologien und Geräten zu klären.
\subsection{Aufgabenstellung}
	Der Markt für technische Hilfsmittel für Diabetiker wächst und die Technologie in der Medizin entwickelt sich enorm weiter. Aber wie kann eine technische Hilfe gestaltet werden, um die Lebensqualität eines Patienten effizient und effektiv zu verbessern? Und wie sind dabei eine gesunde Ernährung und der Einsatz von Sport möglich? Wie kann der Austausch zwischen Diabetikern verbessert werden?\\
	Um diese Fragen beantworten zu können, soll ein System entwickelt werden, welches die Lebensqualität eines Diabetikers steigert. Hierbei sollen zunächst in den Anwendungsbereichen „Ernährung“, „Sport“ und „Kommunikation“ Recherchen und Analysen durchgeführt und auf deren Basis die Prozess- und Systemmodellierung vorgenommen werden. Abschließend gilt es ein System zu  entwickeln und dieses für den Markt konkurrenzfähig fertigzustellen. \newline
	Die Prozessmodellierung dient zur Optimierung des Interface-Designs während der Entwicklung. Mit strukturierten Methoden werden Benutzer und ihre Aufgaben analysiert, Design-Richtlinien angefertigt und User-Interface-Entwürfe des zukünftigen Systems designt. 
	In der Systemmodellierung werden Systemkomponenten und -eigenschaften definiert und die Systemarchitektur modelliert.

\subsection{Vorgehensweise}
	Im Rahmen der Modulen „Entwicklungsprojekte interaktiver Systeme“ im Wintersemester 2018/2019 und „Praxisprojekt“ im Sommersemester 2019 wurde bereits Recherchen und Analysen durchgeführt und ein erster Prototyp eines potentiellen Systems implementiert. Hierbei galt die Konzentration der Konzipierung, Modellierung und Implementierung eines Prototypens, welcher sich auf die Dokumentation des Diabetes-Tagebuches stützt. Im Folgenden wurde eine Evaluation durchgeführt, um die Erkenntnisse und Recherche zu erweitern und eine Grundlage für eine iterative Weiterentwicklung des Prototyps zu schaffen. In Rahmen der Bachelorarbeit im Fachbereich Medieninformatik an der Technischen Hochschule in Köln soll, in einem Zeitraum von 9 Wochen, diese iterative Weiterentwicklung mithilfe der bisherigen Ergebnissen durchgeführt werden. \\
	Die Bachelorarbeit umfasst im wesentlichen das Konzept, die Prozess- und Systemmodellierung, sowie die abschließende Installation. Es gilt, bisherige Artefakte aus den vorherigen Entwicklungsphasen zu überarbeiten und auf diese aufbauend ein fertiges System zu implementieren. Dabei werden die notwendigen Artefakte aus den vorherigen Entwicklungsphasen, welche nicht zu überarbeiten sind, im Anhang beigefügt. Arbeitsprozesse, welche auf Grund des neuen Nutzungskontextes für diese Arbeit nicht brauchbar sind, müssen überarbeitet und/oder erweitert werden. 
