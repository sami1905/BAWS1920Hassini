\section*{Einleitung}\markboth{Einleitung}{Einleitung}\addcontentsline{toc}{section}{Einleitung}
\subsection{Problemstellung}
	Diabetes mellitus stammt aus dem Altgriechischen und bedeutet wörtlich \glqq honigsüßer Durchfluss\grqq{}. Es bedeutet \glqq viel süß schmeckenden Urin\grqq{}. Die Stoffwechselstörung ist seit der Antike bekannt, aber alle Ursachen waren unbekannt und die Behandlung war unmöglich.\cite{SG}\\
	Diabetes ist eine Stoffwechselstörung, bei der der Körper selbst kein Insulin mehr produzieren kann, auch weil der menschliche Organismus dagegen resistent wird. Insulin ist ein Hormon und wird benötigt, um Zucker aus der Blutbahn in die Muskelzellen zu transportieren und den Blutzuckerspiegel zu regulieren. Der Zuckeranteil im Blut ist gesund, wenn sein Wert überwiegend zwischen 60 mg/dL und 140 mg/dL (3,4-7,8 mmol/L) liegt. Der Zucker gelangt nach dem Essen und dem Abbau der Kohlenhydrate in Glukose in Blut und Leber. Wenn die Muskelversorgung mit Zucker gestört ist, bleibt der Zucker im Blut und es beginnt zu übersäuern. Es besteht die Gefahr einer Ketoazidose. Infolgedessen gelangen Ketone in die Blutbahn und in den Urin, wodurch Organe und Körperteile geschädigt werden. Bei Nichtbehandlung kommt es zum Koma oder sogar zum Tod. Insulin wurde erstmals 1922 erfolgreich in einen Menschen injiziert. Erst 1996 kamen sogenannte künstliche Insuline auf den Markt, die bis heute die Grundlage für die Behandlung von Diabetes mellitus bilden. \cite{SG}\\
	Diabetes ist bis heute eine rätselhafte Krankheit, da viele Fragen offen bleiben und vor allem die Ursache der Komplikationen ungeklärt bleibt. Die Prävalenz von Diabetes nimmt stetig zu. Laut der International Diabetes Federation (im Folgenden IDF) lag die Zahl der Menschen mit Diabetes mellitus im Alter zwischen 20 und 79 Jahren im Jahr 2017 bei knapp 425 Millionen oder 8,8\% der Weltbevölkerung. Vergleicht man die Prävalenz von 2017 mit der von 1980, hat sich die Zahl der Diabetiker fast vervierfacht.\cite[S. 9]{IDF}\\
	In Europa hatten im Jahr 2017 rund 6,8\% oder 58 Millionen der 20- bis 79-Jährigen Diabetes, und die IDF prognostiziert im Jahr 2045 rund 67 Millionen Erkrankungen, was einem Anstieg von 16\% entspricht. In Deutschland leben rund 8,3\% der Bevölkerung zwischen 20 und 79 Jahren mit der Stoffwechselkrankheit. Die IDF schätzt, dass weitere 212,4 Millionen Fälle weltweit und 22 Millionen in Europa im Jahr 2017 nicht diagnostiziert wurden.\cite[S. 110 ff.]{IDF} Die Zahl der Todesfälle durch Diabetes mellitus zeigt, dass er als Todesursache unterschätzt wird. Laut IDF sterben weltweit rund 4 Millionen Menschen an Diabetes, davon rund 477,7 Tausend in Europa und 40,2 Tausend in Deutschland.\cite[S. 46]{IDF} Diese Zahlen zeigen, dass diese Stoffwechselstörung trotz des wissenschaftlichen und technologischen Fortschritts in der Medizin zahlreiche Komplikationen mit sich bringt und das Leben der Betroffenen extrem beeinträchtigt. Neben dem Risiko schwerer Folgeerkrankungen sind überwältigende Alltagssituationen, Ernährungsprobleme und mangelnder oder fehlender Sport die Ursachen für die sinkende Lebensqualität der Kranken.\\
	Während der Diabetiker aufgrund von Überzuckerungen (Hyperglykämie) bzw. Unterzuckerungen (Hypoglykämie) jedes Mal gezwungen wird, sportliche Aktivitäten einzustellen, verursachen falsch berechnete Insulineinheiten während der Therapie schlechte Blutzuckerwerte. Oft mangelt es an Möglichkeiten, Erfahrungen unter Diabetikern auszutauschen und Fragen zu neuen Behandlungsmethoden oder neuen Technologien und Geräten zu klären.
\subsection{Aufgabenstellung}
	Ziel dieses Projekts ist es in erster Linie, den Anwender bei der Behandlung seines Diabetes zu unterstützen und seine Lebensqualität zu verbessern. Da es sich hierbei um die Weiterentwicklung eines Systems handelt und bereits eine Entwicklungsphase abgeschlossen ist, wurde zu Beginn der ersten Entwicklungsphase die Zielhierarchie des Projekts erstellt. Dies ist im Anhang ab Seite ab Seite \pageref{section:Zielhierarchie} ersichtlich. \\
	Der Markt für technische Hilfsmittel für Diabetiker wächst und die Technologie in der Medizin entwickelt sich enorm. Aber wie kann technische Hilfe gestaltet werden, um die Lebensqualität eines Patienten effizient und effektiv zu verbessern? Und wie sind gesunde Ernährung und Sport möglich? Wie kann der Austausch zwischen Diabetikern verbessert werden?\\
	Um diese Fragen zu beantworten, sollte ein bereits entwickeltes System, das die Lebensqualität eines Diabetikers erhöht, weiterentwickelt werden. Zunächst sollen Forschungen und Analysen in den Anwendungsbereichen \glqq Ernährung\grqq{}, \glqq Sport\grqq{} und \glqq Kommunikation\grqq{} durchgeführt und darauf aufbauend die Benutzer- und Aufgabenmodellierung durchgeführt werden. Abschließend gilt es die Benutzeroberfläche zu entwerfen, das System zu modellieren und es für den Markt wettbewerbsfähig zu machen.
\subsection{Vorgehensweise}
	Im Rahmen der Bachelorarbeit im Bereich Medieninformatik an der Technischen Universität zu Köln soll die iterative Weiterentwicklung anhand der bisherigen Ergebnisse über einen Zeitraum von 9 Wochen durchgeführt werden.\\
	In den Modulen \glqq Entwicklungsprojekte interaktiver Systeme\grqq{} im Wintersemester 2018/2019 und \glqq Praxisprojekt\grqq{} im Sommersemester 2019 wurde bereits geforscht, analysiert und ein erster Prototyp eines möglichen Systems implementiert. Im ersten Konzept wurde die Domäne des Diabetes mellitus vorgestellt und dessen Markt analysiert. Dies wurde verwendet, um die ersten Alleinstellungsmerkmale und Anforderungen für ein potenzielles System zu definieren. Um das Themenfeld zu erweitern und auch die neuen Anwendungsbereiche analysieren zu können, wird eine neue Recherche durchgeführt. Da die erste Entwicklungsphase über ein Jahr zurückliegt, muss die Marktrecherche auf den neuesten Stand gebracht werden. Im methodischen Rahmen wurde in der Vergangenheit das Prozessmodell nach Mayhews als Leitlinie des Projekts gewählt. Da Mayhews Usability Engineering Lifecycle einen iterativen Entwicklungsprozess ermöglicht, werden ausgewählte Entwicklungsschritte im Rahmen dieser Arbeit überarbeitet oder neu ausgerichtet. Die Modellierung von Benutzern und Aufgaben ist für diese Arbeit von grundlegender Bedeutung, da der Fokus auf dem Benutzer und den Prozessen seiner Arbeit liegt. Diese müssen in Bezug auf die Anwendungsbereiche modelliert und analysiert werden, damit Standards und Konventionen für die Auslegung und das Design des Systems festgelegt werden können. Abschließend muss das System modelliert, beschrieben und implementiert werden. Eine bereits durchgeführte Evaluation (S. \pageref{section:Evaluation}) und die daraus gewonnenen Erkenntnisse sollten berücksichtigt und in die Gesamtneuausrichtung des Systems einbezogen werden.\\
	Im Wesentlichen umfasst die Bachelorarbeit Teile eines Konzepts, die Benutzer- und Aufgabenmodellierung sowie die Beschreibung der Systemstrukturen und die abschließende Implementierung. 

