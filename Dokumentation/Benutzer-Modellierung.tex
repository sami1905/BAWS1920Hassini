\section{Benutzermodellierung}
	Der Usability Engineering Lifecycle nach Mayhew hilft, die Benutzerfreundlichkeit eines Systems (Usability) und das entsprechende Design der Benutzeroberflächen (User Interfaces) zu erreichen. Nach Mayhew wird diese Benutzerfreundlichkeit daran gemessen, wie einfach eine Benutzeroberfläche verwendet und erlernt werden kann. Die Benutzermodellierung ist Teil der „Requirements Analysis“, eine von drei fundamentalen Prozess-Phasen im Usability Engineering Lifecycle und dient zur Anforderungsanalyse. \cite{MD}
	\subsection{Stakeholder-Analysis}
	Um für das zu entwickelnde System eine strukturierte Analyse von Usability-Anforderungen durchführen zu können, muss zunächst die Zielgruppe der Anwerdung bestimmt werden. Zwar lässt sich diese bereits in den früheren Entwicklungsphasen und auch während der erneuten Themenfeldanalyse gut definieren, allerding sind diese Benutzergruppen und ihr Bezug zur Domäne und zum System und  seinen Merkmalen in dieser Phase noch zu überarbeiten und anhand  der Erwartungen und Erfordernisse spezifischer zu definieren.  Mit Hilfe der Stakeholder-Analyse (Tabelle  \ref{tab:Stakeholder}: \nameref{tab:Stakeholder}) werden die Benutzer anhand ihrer Erwartungen und Anforderungen in Kategorien eingeteilt und diese konsequent auf Konflikte hin analysiert.
	\begin{center}
		\begin{longtable}[H]{|p{3cm}|p{2cm}|p{4cm}|p{4.5cm}|}
			\hline
			\textbf{Bezeichnung} & \textbf{Bezug} & \textbf{Objektbereich} & \textbf{Erfordernis/Erwartung}\\
			\hline
			Diabetiker & Anrecht & System & ein Hilfsmittel für den Umgang mit Diabetes\\
			\cline{2-4}
			& Anteil & Merkmal: Datensicherung & von Benutzer eingegebene Daten werden sicher verwaltet\\
			\cline{3-4}
			& & Merkmal: Funktionen zum Tausch von Erfahrungen & um einen Erfahrungsaustausch zwischen Diabetikern zu ermöglichen, sind Erfahrungen von verscheidenen Benutzern essentiell\\
			\cline{2-4}
			& Anspruch & Merkmal: Funktionen zur Dokumentation von Diabetes, Ernährung und Aktivität & ausführliche Dokumenation muss gewährleistet sein\\
			\cline{3-4}
			& & Merkmal: Funktion zur Kontaktaufnahme zu anderen Diabetikern & sozialer Kontakt zu anderen Diabetikern muss gewährleistet sein\\
			\cline{3-4}
			& & Merkmal: Berechnung von Daten & Berechnungen von individuellen Daten müssen gewährleistet sein und korrekt durchgeführt werden\\
			\cline{3-4}
			\newpage
			\cline{3-4}
			& & Merkmal: user interface & Benutzung selbsterklärend, einfach zu lernen, zu merken\\
			\cline{2-4}
			& Interesse & System & Steigerung des Wohlbefindens und der Lebensqualität\\
			\hline
			Personen aus gemeinsamen Haushalt/Umfeld & Anrecht & System & ein Hilfsmittel für die Unterstützung bei Behandlung\\
			\cline{2-4}
			& Anteil & - & - \\
			\cline{2-4}
			& Anspruch & Merkmal: Funktionen zur Berechnung von Nährwerten eines Lebensmittels & Personen (Eltern, Lebenspartner, etc.), die für einen Diabetiker kochen, sollten keine Nährwerte zählen und berechnen müssen\\
			\cline{3-4}
			& & Merkmal: Zugriff auf Blutzuckerdaten eines ausgewählten Diabetikers & die persönlichen Blutzuckerwerte eines Diabetikers sollten einzusehen sein\\
			\cline{3-4}
			& & Merkmal: user interface & Benutzung selbsterklärend, einfach zu lernen, zu merken\\
			\cline{2-4}
			& Interesse & System & Steigerung des Wohlbefindens und der Lebensqualität\\
			\hline 
			Arzt & Anrecht & System & vereinfachte Darstellung der Blutzuckerwerte zur Analyse\\
			\cline{2-4}
			& Anteil & Merkmal: Datensicherung & vom Arzt festgelegte Faktoren können individuell eingespeichert werden\\
			\cline{2-4}
			& Anspruch &  Merkmal: Zugriff auf Blutzuckerdaten eines ausgewählten Diabetikers & die persönlichen Blutzuckerwerte eines Diabetikers sollten einzusehen sein\\
			\cline{3-4}
			& & Merkmal: user interface & Benutzung selbsterklärend, einfach zu lernen, zu merken\\
			\cline{2-4}
			& Interesse & System & Steigung des Wohlbefindens und der Lebensqualität der Patienten\\
			\hline
			\newpage
			\hline
			Krankenkassen & Anrecht & - & -\\
			\cline{2-4}
			& Anteil & System & Übernahme eventueller Kosten für die Nutzung des Systems\\
			\cline{2-4}
			& Anspruch & System & ein finanzierbares System\\
			\cline{2-4}
			& Interesse & System & Patienten bevorzugen Krankenkassen mit einer großteiligen Übernahme der Kosten des Systems\\
			\hline
			Pharmaindustie & Anrecht & - & -\\
			\cline{2-4}
			& Anteil & - & -\\
			\cline{2-4}
			& Anspruch & Medikamenten & Profit durch Verkauf von Medikamenten\\
			\cline{2-4}
			& Interesse & Insulinbedarf & mehr Insulinbedarf der Patienten bedeutet mehr Profit\\
			\hline
			Konkurrenz & Anrecht & - & -\\
			\cline{2-4}
			& Anteil & - & -\\
			\cline{2-4}
			& Anspruch & - & -\\
			\cline{2-4}
			& Interesse & Verkauf von eigenem Produkt & hohe Verkaufszahlen des eigenen Produkts und niedrige Verkaufszahlen der Konkurrenzprodukte\\
			\hline
			Stores für mobile Anwendungen & Anrecht & - & -\\
			\cline{2-4}
			& Anteil & - & -\\
			\cline{2-4}
			& Anspruch & - & -\\
			\cline{2-4}
			& Interesse & Verkauf von Produkt & Profit durch Erwerb des Systems\\
			\hline
			\captionsetup{justification=centering}
			\caption{Stakeholder-Analysis}
			\label{tab:Stakeholder}
		\end{longtable}
	\end{center}
	\setlength{\parindent}{0pt}Mit Hilfe der Tabelle \ref{tab:Stakeholder}: \nameref{tab:Stakeholder} können Diabetiker und Personen aus ihrem Umfeld wie Eltern oder Lebenspartner eindeutig als Zielgruppe des zu entwickelnden Systems identifiziert werden. Die Tabelle zeigt jedoch nicht, dass es zwei verschiedene Arten von Diabetikern gibt (Typ-1- und Typ-2-Diabetiker). Obwohl die beiden Benutzerkategorien dieselben Anforderungen haben, muss eine Unterscheidung nach der Wichtigkeit der Anforderungen getroffen und an das System gestellt werden: Bei Typ-1-Diabetes ist aufgrund der Notwendigkeit einer Insulintherapie die Dokumentation von diabetesrelevanten Daten wie Blutzuckerspiegel, Insulin und Broteinheiten unerlässlich, während bei den Typ-2-Diabetikern Insulinmangel nicht immer die Ursache für eine Erkrankung ist. Vielmehr sind hier eine gesunde Ernährung und regelmäßige Bewegung entscheidend für die Regression dieser Art von Diabetes und die Aufzeichnung und Dokumentation von Mahlzeiten und sportlichen Aktivitäten hat daher einen höheren Stellenwert als die Dokumentation von Insulineinheiten. Aus diesem Grund müssen die verschiedenen Benutzerkategorien bei der Weiterentwicklung berücksichtigt werden.\\
	Neben den Stakeholdern mit ihren positiven Erwartungen an das System oder seine Funktionen gibt es weitere Stakeholder, die möglicherweise in Konflikt mit dem System stehen. Die Pharmaindustrie und Apotheken erzielen Umsatz durch den Verkauf von Medikamenten. Da das zu entwickelnde System den Blutzuckerspiegel konstant regulieren und die Rückbildung des Typ-2-Diabetes bewirken sollte,  hätte die Pharmaindustrie kein großes Interesse an der Entwicklung eines solchen Systems.  Eine Möglichkeit, diesen Interessenkonflikt zu lösen, könnte darin bestehen, Apotheken als Verkaufs- oder Marketingfläche zu nutzen. Die Apotheken könnten durch Werbung und Zusammenarbeit Gewinne ebenfalls erzielen, und die Pharmaindustrie könnte als Kooperationspartner für das Entwicklungsteam fungieren. Wettbewerberunternehmen haben logischerweise auch kein Interesse an der Entwicklung des Systems, da sie den Verkauf ihrer eigenen Produkte den Produkten anderer Unternehmen vorziehen. Auch hier wäre eine Kooperation eine Option zur Konfliktlösung und zur Erweckung von Interessen. Technologien wie Sensoren oder Geräte von Wettbewerbern zu Erfassung von Blutzuckerwerten könnten  über Programmierschnittstellen in Systeme von Drittanbietern mit Gewinnbeteiligung integriert werden. Dies erleichtert die Erfassung von Blutzuckerdaten und macht die Herstellung zusätzlicher Messgeräte überflüssig. Der Wettbewerb generiert auch zusätzliche Gewinne durch die Entwicklung des Systems und die Bereitstellung eigener Schnittstellen.	
	\subsection{User-Profiles}
	Um im weiteren Entwicklungsverlauf die Anforderungen der verschiedenen Benutzerkategorien an das System zu erhalten, werden im Folgenden User Profiles für die definierten Kategorien Typ-1-Diabetiker (Tabelle \ref{tab:User-Profile-1}) und Typ-2-Diabetiker (Tabelle \ref{tab:User-Profile-2}) erstellt. Beim Anlegen dieser User Profiles können Benutzereigenschaften und -merkmale auch aus der bereits durchgeführten Befragung von Probanden aus den beiden Benutzerkategorien (s. Anhang: A \nameref{section:Evaluation} ab Seite \pageref{section:Evaluation}) definiert werden. Diese Merkmale sind wie folgt zu ordnen:
	\begin{itemize}
		\item Physische Merkmale
		\item Psychologische Merkmale
		\item Wissen und Erfahrungen
		\item Aufgabenmerkmale
	\end{itemize}
	Alternativ können Diabetiker Gruppen auch in „Kinder und Jugendliche mit Diabetes“ und „Erwachsene mit Diabetes“ eingeteilt werden. Ergebnisse der bisherigen Recherche zeigen jedoch, dass eine Aufschlüsselung nach Diabetes-Typen eine höhere Priorität hat als die nach Altersgruppen. 
	\subsubsection{Typ-1-Diabetiker}
	\begin{center}
		\begin{longtable}[H]{p{6.6cm}p{6.6cm}}
			\multicolumn{2}{c}{User Profile - Typ-1-Diabetiker} \\
			\toprule
			\multicolumn{2}{l}{User Category Identifiers}\\
			\multicolumn{2}{p{13.6cm}}{Menschen zwischen 5 und 82 Jahren mit Typ-1-Diabetes aus Deutschland kommen nicht aus einer definierten Arbeitsgruppe und haben Keine Erfahrung mit dem Einsatz mobiler Anwendungen.} \\\\
			\textbf{Merkmal} & \textbf{Merkmalsausprägung}\\
			\midrule
			1. Physische Merkmale & \\[.5\normalbaselineskip]
			Geschlecht & \tabitem männlich\\
			& \tabitem weiblich \\ 
			& \tabitem diverse \\[.3\normalbaselineskip]
			Alter & 5-82 Jahre \\[.3\normalbaselineskip]
			Händigkeit & Links- und Rechtshänder \\[.3\normalbaselineskip]
			Wohnort & national (Deutschland)\\[.3\normalbaselineskip]
			Gesundheitszustand & \tabitem Diabetes mellitus Typ 1\\
			& \tabitem Folgeerkrankungen\\
			& \tabitem körperliche Behinderung\\[.3\normalbaselineskip]
			Sozio-ökonomischer Status &  \tabitem Schüler/-in\\
			& \tabitem Studierende-/r \\ 
			& \tabitem Auszubildende-/r\\
			& \tabitem Angestellte-/r\\
			& \tabitem Ausgelehrte./r\\
			& \tabitem Arbeitslose-/r\\
			& \tabitem Berufliche Selbstständigkeit\\[.3\normalbaselineskip]
			Einkommen & \tabitem kein Einkommen\\
			& \tabitem Taschengeld\\
			& \tabitem geregeltes/staatliches Einkommen\\[.3\normalbaselineskip]
			\midrule
			2. Psychologische Merkmale & \\[.5\normalbaselineskip]
			Behandlungsart & \tabitem Insulinspritzen\\
			& \tabitem Insulinpumpe\\[.3\normalbaselineskip]
			Behandlungsziele & \tabitem regulierte Blutzuckerwerte\\
			& \tabitem Vermeidung der Folgeerkrankungen\\
			& 	\tabitem hoche Lebensqualität\\[.3\normalbaselineskip]
			Lebensziele & \tabitem Schul-/Studium-/Ausbildungsabsch-lüsse\\
			& \tabitem Steigung des Wohlbefindens und der Lebensqualität\\
			& \tabitem Weiterbildung\\
			& \tabitem Existenzsicherheit\\
			& \tabitem möglichst lange Lebenszeit\\
			& \tabitem beruflicher Aufstieg\\
			& \tabitem hohe Lebensqualität\\[0.3\normalbaselineskip]
			Motivation zu Nutzung des zukünftigen Systems & \tabitem einfache Handhabung der Diabetes\\
			& \tabitem besser Blutzuckerwerte\\
			& \tabitem keine analoge/schnelle Dokumentation\\
			& \tabitem Abnahme der BE-Berechnung\\
			& \tabitem Abnahme der Insulinberechnung\\
			& \tabitem zeitsparende Anwendungen\\
			& \tabitem Darstellung der Blutzuckerwerte in verständlicher Formen (Graphen oder Tabelle)\\
			& \tabitem Risiko der Folgeerkrankungen reduzieren\\
			& \tabitem Stressreduzierung\\[0.3\normalbaselineskip]
			\midrule
			3. Wissen und Erfahrungen  & \\[.5\normalbaselineskip]
			Erfahrung im Anwendungsgebiet & ausreichend bis sehr gut, durch Schulungen und Eigenbehandlung des Diabetes seit Erkrankung\\[.3\normalbaselineskip]
			Technische Unterstützung bei Therapie & \tabitem Blutzuckermessgerät\\
			& \tabitem Insulinpumpe \\[0.3\normalbaselineskip]
			\midrule
			4. Aufgabenmerkmale & \\[.5\normalbaselineskip]
			Kenntnisse und Fähigkeiten & \tabitem Lesen/Schreiben/Rechnen\\
			& \tabitem Berechnung von Kohlenhydrat, BE's und Insulineinheiten\\
			& \tabitem Diabetes-Dokumentation\\ 
			& \tabitem Nutzung von verfügbaren Technologien\\[0.3\normalbaselineskip]
			Verfügbare relevante Technologien & \tabitem Smartphones\\
			& \tabitem Smartwatches\\
			& \tabitem Tablets\\[0.3\normalbaselineskip]
			
			\bottomrule
			\captionsetup{justification=centering}
			\caption{User Profile: Typ-1-Diabetiker}
			\label{tab:User-Profile-1}
		\end{longtable}
	\end{center}
	\subsubsection{Typ-2-Diabetiker}
	\begin{center}
		\begin{longtable}[H]{p{6.6cm}p{6.6cm}}
			\multicolumn{2}{c}{User Profile - Typ-2-Diabetiker} \\
			\toprule
			\multicolumn{2}{l}{User Category Identifiers}\\
			\multicolumn{2}{p{13.6cm}}{Menschen aus Deutschland, von denen die meisten im Erwachsenenalter an Typ-2-Diabetes erkrankt sind, in den letzten Jahren auch zunehmend im Jugendalter, kommen aus keiner definierten Arbeitsgruppe und haben Erfahrung mit dem Einsatz mobiler Anwendungen.} \\\\
			\textbf{Merkmal} & \textbf{Merkmalsausprägung}\\
			\midrule
			1. Physische Merkmale & \\[.5\normalbaselineskip]
			Geschlecht & \tabitem männlich\\
			& \tabitem weiblich \\ 
			& \tabitem diverse \\[.3\normalbaselineskip]
			Alter & 16-82 Jahre \\[.3\normalbaselineskip]
			Händigkeit & Links- und Rechtshänder \\[.3\normalbaselineskip]
			Wohnort & national (Deutschland)\\[.3\normalbaselineskip]
			Gesundheitszustand & \tabitem Diabetes mellitus Typ 1\\
			& \tabitem Folgeerkrankungen\\
			& \tabitem körperliche Behinderung\\
			& \tabitem Fettleibigkeit\\
			& \tabitem schlechter Ernährungsstil\\[.3\normalbaselineskip]
			Sozio-ökonomischer Status &  \tabitem Schüler/-in\\
			& \tabitem Studierende-/r \\ 
			& \tabitem Auszubildende-/r\\
			& \tabitem Angestellte-/r\\
			& \tabitem Ausgelehrte./r\\
			& \tabitem Arbeitslose-/r\\
			& \tabitem Berufliche Selbstständigkeit\\[.3\normalbaselineskip]
			Einkommen & \tabitem kein Einkommen\\
			& \tabitem Taschengeld\\
			& \tabitem geregeltes/staatliches Einkommen\\[.3\normalbaselineskip]
			\midrule
			2. Psychologische Merkmale & \\[.5\normalbaselineskip]
			Behandlungsart & \tabitem Tabletten\\
			& \tabitem Diät\\
			& \tabitem bei Bedarf Insulintherapie\\[.3\normalbaselineskip]
			Behandlungsziele & \tabitem Körpergewichtreduzierung\\
			& \tabitem ausgewogene Ernährung\\
			& \tabitem regulierte Blutzuckerwerte\\
			& \tabitem Vermeidung der Folgeerkrankungen\\
			& 	\tabitem hoche Lebensqualität\\[.3\normalbaselineskip]
			Lebensziele & \tabitem Schul-/Studium-/Ausbildungsabsch-lüsse\\
			& \tabitem Steigung des Wohlbefindens und der Lebensqualität\\
			& \tabitem Weiterbildung\\
			& \tabitem Existenzsicherheit\\
			& \tabitem möglichst lange Lebenszeit\\
			& \tabitem beruflicher Aufstieg\\
			& \tabitem hohe Lebensqualität\\[0.3\normalbaselineskip]
			Motivation zu Nutzung des zukünftigen Systems & \tabitem einfache Handhabung der Diabetes\\
			& \tabitem Reduzierung des Körpergewichts\\
			& \tabitem Rückbildung der Erkrankung\\
			& \tabitem stabile Blutzuckerwerte\\
			& \tabitem keine analoge/schnelle Dokumentation\\
			& \tabitem zeitsparende Anwendungen\\
			& \tabitem Risiko der Folgeerkrankungen reduzieren\\
			& \tabitem Stressreduzierung\\[0.3\normalbaselineskip]
			\midrule
			3. Wissen und Erfahrungen  & \\[.5\normalbaselineskip]
			Erfahrung im Anwendungsgebiet & keine bis sehr gut, durch ärztliche Unterstützung und Eigenbehandlung des Diabetes seit Erkrankung\\[.3\normalbaselineskip]
			Technische Unterstützung bei Therapie & \tabitem Blutzuckermessgerät\\[0.3\normalbaselineskip]
			\midrule
			4. Aufgabenmerkmale & \\[.5\normalbaselineskip]
			Kenntnisse und Fähigkeiten & \tabitem Lesen/Schreiben/Rechnen\\
			& \tabitem Diabetes-Dokumentation\\ 
			& \tabitem Nutzung von verfügbaren Technologien\\[0.3\normalbaselineskip]
			Verfügbare relevante Technologien & \tabitem Smartphones\\
			& \tabitem Smartwatches\\
			& \tabitem Tablets\\[0.3\normalbaselineskip]
			\bottomrule
			\captionsetup{justification=centering}
			\caption{User Profile: Typ-2-Diabetiker}
			\label{tab:User-Profile-2}
		\end{longtable}
	\end{center}
	Es empfiehlt sich, in der Benutzermodellierung auf der Grundlage des Benutzerprofils sogenannte Personas zu entwerfen, um reale Personen in einer fiktiven Darstellung in die Modellierung einzubeziehen. Zwar ist dies im Usability Engineering Lifecycle nicht vorgegeben und optional, allerdings bieten Personas eine Grundlage für die spätere Aufgabenmodellierung. Es wird folglich jeweils ein Persona zum User Profil der Typ-1 und der Typ-2-Diabetiker angefertigt. Die erste Persona wird in Anlehnung der Dokumentation von Journal Stuttgart - RegioTV „Leben mit Diabetes“ vom 15.08.2018 und die zweite in Anlehung der Dokumentation von NDR Ratgeber „Typ-2-Diabetes: Wie man vom Insulin wieder wegkommt | Die Ernährungs-Docs | NDR“ verfasst. 
	\subsection{Personas}
	\textbf{Persona - Typ-1-Diabetiker}
	\begin{center}
		\begin{longtable}[H]{p{6.6cm}p{6.6cm}}
			\addtocounter{table}{-1}
			\texttt{Name: }& \texttt{Lucie}\\
			\texttt{Alter: }& \texttt{6 Jahre}\\
			\texttt{Geschlecht: }&\texttt{weiblich}\\
			\texttt{Wohnort:} & \texttt{Stuttgart, bei ihren Eltern}\\
			\texttt{Gesundheitszustand:} & \texttt{Diabete-mellitus-Typ-1}\\
			\texttt{Sozio-ökonomischer Status: }& \texttt{Schülerin}\\
			\texttt{Behandlungsart:} & \texttt{Insulinpumpe}\\
			&  \texttt{stündliche Blutzuckermessung}\\
			&  \texttt{analoges Tagebuch}\\
			\texttt{Behandlungsziel:} &  \texttt{Diabetes kontrollieren}\\
			&  \texttt{ein normales Leben}\\
			\texttt{Erfahrungen: }&  \texttt{seit 3 Jahren an Diabetes erkrankt}\\
			&  \texttt{Erfahrung der Eltern und Betreuer durch Schulungen in Krankenhäuser}\\
			\texttt{Technische Unterstützung:} & \texttt{FreeStyle-Libre Blutzuckermessgerät}\\
			&  \texttt{Insulinpumpe}\\
		\end{longtable}
	\end{center}
	\texttt{Lucie ist 6 Jahre alt und seit 3 Jahren an Diabetes mellitus Typ 1 \newline
		erkrankt. Gleich am Morgen wird Lucie mit ihrer Krankheit konfrontiert, \newline 
		denn ein Leben wie ein gesundes Kind wird sie nicht mehr leben können. \newline
		Noch vor dem Frühstück misst sie gemeinsam mit ihrer Mutter ihren \newline
		Blutzuckerspiegel. Fast alle  Aufgaben in ihrer Diabetesbehandlung macht \newline 
		sie gemeinsam mit ihren Eltern. Ihre Mutter musste nach der Erkrankung \newline
		von Lucie ihre Arbeit kündigen, um jederzeit abrufbereit zu sein.}\newline
		 \texttt{Lucie schnappt sich ihr Blutzuckermessgerät und legt es auf ihren\newline
		 	 FreeStyle Libre-Sensor am Oberarm. Ihr Blutzuckerspiegel ist in einem \newline 
		 	 gesunden Bereich. Als es an den Frühstückstisch geht, muss Lucies Mutter \newline
		 	 sie an das Spritzen des Insulins für die Mahlzeit erinnern. Gemeinsam \newline
		 	 schnappen sie sich die  Wage, wiegen die Scheibe Vollkornbrot \newline
		 	 und den Apfel und ermitteln anhand der Kohlenhydrate die Broteinheiten\newline
		 	  und Insulineinheit, die Lucie spritzen muss. Sie zückt ihre \newline
		 	 Insulinpumpe und trägt die Insulineinheiten ein, die ihre Mutter \newline
		 	 ihr vorgerechnet hat. Aber auch nach dem Spritzen des Insulins muss \newline 
		 	 sich Lucie noch einige Minuten gedulden. Denn wenn sie direkt essen \newline
		 	 würde, würde sie den nötigen Spritz-Ess-Abstand von 10 Minuten nicht \newline
		 	 einhalten und  die Kohlenhydrate des Essens würden schneller wirken \newline
		 	 als das Insulin. }\newline
		\texttt{Nachdem Lucie ihr Frühstück verspeist und ihre Schulsachen \newline
			zusammengepackt hat, fährt Lucies Mutter sie in die Schule. \newline
			Lucie besucht eine ganz besondere Schule. Sie besucht die Waldschule \newline
			in Stuttgart. Die Waldschule ist die erste Schule, die Schulklassen aus \newline
			Schülern mit Diabetes gebildet hat. Neben diesen Klassen gehen auch \newline
			gesunde Schüler zur Waldschule. Lucies Eltern sind froh, dass es diese \newline
			 Schule gibt, auch wenn sie monatlich 160\euro für diese zahlen  müssen. \newline
			 Bereits bei der Suche eines Kitaplatzes hatten Lucies Eltern große \newline
			 Probleme und erhielten auf Nachfrage einige Absagen. An der Waldschule \newline
			 werden die Schüler von geschulten Lehrer unterrichtet und von \newline
			 Diabetesassistenten und Krankenschwestern betreut, die Fachwissen\newline
			  besitzen und immer mit nötiger Sicherheit in Situation handeln können. \newline
			  Dennoch ist Lucie auch während der Schulzeit und ihrer Freizeit \newline
			  dazu verpflichtet, den Diabetes zu kontrollieren. Zur Sicherheit trägt sie \newline
			  immer eine kleine Tasche mit Süßigkeiten, Blutzuckermessgerät\newline
			   und Notfallspritzen bei sich mit. Blutzuckermessungen  führt sie \newline
			   stündlich durch. Während den großen Pausen und dem Sportunterricht besteht \newline
			   wegen der Bewegung die Gefahr eines niedrigen Blutzuckerspiegels. Hierzu \newline
			   misst Lucie sogar alle 20-30 Minuten ihren Blutzucker und isyt vor dem Sport \newline
			   eine Banane oder Traubenzucker. Wenn sie während der Schulzeit ihren \newline
			   Blutzuckerspiegel misst, trägt sie diesen und auch ihre Mahlzeiten in ihr \newline
			   analoges Tagebuch ein, damit ihre Eltern nach der Schule diese kontrollieren können.\newline
			    Lucie kann ihre Erkrankung und ihre Blutzuckerwerte schon sehr gut \newline
			    einschätzen, allerding wünscht sie sich sehr oft, einfach ein normales \newline
			    Leben leben zu können. }\\
		    
		    
		    \textbf{Persona - Typ-2-Diabetiker}
		    \begin{center}
		    	\begin{longtable}[H]{p{6.6cm}p{6.6cm}}
		    		\addtocounter{table}{-1}
		    		\texttt{Name: }& \texttt{Bernd Pache}\\
		    		\texttt{Alter: }& \texttt{53 Jahre}\\
		    		\texttt{Geschlecht: }&\texttt{männlich}\\
		    		\texttt{Wohnort:} & \texttt{Hamburg}\\
		    		\texttt{Gesundheitszustand:} & \texttt{Diabete-mellitus-Typ-2}\\
		    		\texttt{Sozio-ökonomischer Status: }& \texttt{Schulleiter}\\
		    		\texttt{Behandlungsart:} & \texttt{Insulinspritze}\\
		    		&  \texttt{Tabletten}\\
		    		\texttt{Behandlungsziel:} &  \texttt{Insolinlose Behandlung}\\
		    		&  \texttt{Körpergewicht reduzieren}\\
		    		\texttt{Erfahrungen: }&  \texttt{seit 8 Jahren an Diabetes erkrankt}\\
		    		&  \texttt{Schulungen zur Ernährung}\\
		    	\end{longtable}
		    \end{center}
		    \texttt{Bernd Pache ist 53 Jahre alt und leidet seit 8 Jahren an Diabetes mellitus \newline
		    	Typ 2. Da seine Muskelzellen inzwischen eine sehr geringe Insulin-\newline
		    	empfindlichkeit aufweisen und das vom Körper produzierte \newline
		    	Insulin nicht ausreicht, muss der Schulleiter aus Hamburg neben der \newline
		    	Einnahme von Tabletten auch täglich Insulin spritzen. Dieses Insulin \newline
		    	reduziert zwar den Blutzuckerspiegel, sorgt allerdings auch für ein \newline
		    	höheres Risiko dauerhaft das Körpergewicht zuzunehmen. Senkt das \newline
		    	Insulin den Blutzuckerspiegel zu sehr und Bernd unterzuckert, \newline
		    	bekommt er Hunger und muss etwas essen. Isst Bernd dann mehr\newline
		    	als für die Unterzuckerung notwendig war, muss er erneut Insulin spritzen.\newline
		    	Ein Teufelskreis. Nach 4 Jahren Insulinbehandlung und 20 kg \newline
		    	Gewichtszunahme möchte Bernd sein Lebensstil ändern und so \newline
		    	schnell wie möglich das Spritzen von Insulin absetzten. Um dies umsetzten\newline
		    	zu können, gilt es sich ausgewogen zu Ernähren und regelmäßig Sport zu \newline
		    	treiben. Gefährliche Kombinationen aus Kohlenhydrate und Fett darf Bernd \newline
		    	nicht mehr zu sich nehmen. Stattdessen stehen viel Eiweiß, viel Gemüse \newline
		    	und keine Süßigkeiten auf den Speiseplan. Seine Kohlenhydrate-, \newline
		    	Eiweiß- und Fettaufnahme muss Bernd ebenfalls im Auge behalten. \newline
		    	Nach 6 Monaten bewusster Ernährung, aus überwiegend Obst und Gemüse,\newline
		    	und regelmäßigem Sport sind bereits erstaunliche Erfolge zu sehen. \newline
		    	Bernd hat insgesamt 21 kg Körpergewicht	verloren und konnte das \newline
		    	Insulinspritzen bereits nach 2 Monaten einstellen. Auch seine \newline
		    	Blutzuckerwerte haben sich verbessert. Sein HbA1c-Wert ist \newline
		    	von 8\% auf 6,8\% gesunken. Führt Bernd auch weiterhin den Lebensstil \newline
		    	wie in den vergangenen 6 Monaten, ist das Halbieren oder sogar \newline
		    	Absetzen der Tablettendosis das neue Ziel.}
	
	\subsection{Anforderungen}
	Um anhand der User Profiles und der Personas eine kontextbezogene Aufgabenanalyse (Contextual Task Analysis) erstellen und die Nutzung des Systems modellieren zu können, müssen zunächst einige Funktionen des zukünftigen Systems identifiziert werden. Dazu müssen mit Hilfe der Stakeholder-Analysis und der User Profiles Systemanforderungen erstellt werden.
	Die Anforderungen werden durch Modellierung der Benutzer in Form von Stakeholder-Analysen, Benutzerprofilen und Personas beschrieben. Sie lassen sich in funktionale und non-funktionale Anforderungen unterteilen. Die funktionalen Anforderungen müssen den wesentlichen Benutzerkategorien zugeordnet werden, für die User Profiles zuvor erstellt wurden.
	\subsubsection{funktionale Anforderungen}
	\paragraph{Für alle Diabetiker}\mbox{}
	\begin{itemize}
		\item\textbf{\lbrack F10\rbrack} Das System muss dem Benutzer die Möglichkeit bieten, ein individuelles Benutzerkonto anzulegen.
		\item\textbf{\lbrack F20\rbrack}  Das System muss dem Benutzer die Möglichkeit bieten, die individuellen Daten seines Benutzerkontos zu bearbeiten.
		\item\textbf{\lbrack F30\rbrack} Das System muss dem Benutzer die Möglichkeit bieten, sein angelegtes Benutzerkonto wieder zu löschen.
		\item\textbf{\lbrack F40\rbrack}	Das System kann dem Benutzer die Möglichkeit bieten, Blutzuckerwerte die von externen Blutzuckermessgeräten erfasst wurden per API in das System zu übernehmen.
		\item\textbf{\lbrack F50\rbrack}	Das System muss dem Benutzer die Möglichkeit bieten, unterschiedliche Ereignisse (Blutzuckerwert, Mahlzeit und sportliche Aktivität) manuell in das Tagebuch einzutragen.
		\item\textbf{\lbrack F50\rbrack} Das System soll dem Benutzer die Ereignisse und Blutzuckerwerte in Form von Tagebucheinträgen und anhand eines Graphen repräsentieren.
		\item\textbf{\lbrack F70\rbrack} Falls ein Ereignis bereits vorhanden ist, soll das System dem Benutzer die Möglichkeit bieten, diesen zu ändern oder um weitere Daten zu erweitern zu können.
		\item\textbf{\lbrack F80\rbrack} Das System soll dem Benutzer die Möglichkeit bieten, auf eine Lebensmittel-Datenbank zugreifen zu können.
		\item\textbf{\lbrack F90\rbrack} Das System soll dem Benutzer die Möglichkeit bieten, erfasste Aktivitäten durch andere Anwendungen einzupflegen.
		\item\textbf{\lbrack F100\rbrack} Das System muss dem Benutzer die Möglichkeit bieten, Aktivitäten hinzuzufügen.
		\item\textbf{\lbrack F110\rbrack} Das System soll dem Benutzer den Kalorienbedarf, bestehend aus Grundumsatz und Leistungsumsatz, repräsentieren.
		\item\textbf{\lbrack F120\rbrack} Das System muss dem Benutzer die Möglichkeit bieten, Kontakt zu anderen Benutzern in einer Kommunikationsplattform herzustellen.
		\item\textbf{\lbrack F130\rbrack} Das System muss dem Benutzer die Möglichkeit bieten, Beiträge hinzuzufügen, zu entfernen und zu bearbeiten.
		\item\textbf{\lbrack F140\rbrack} Das System soll dem Benutzer Beiträge andere Benutzer repräsentieren.
		\item\textbf{\lbrack F150\rbrack} Das System soll dem Benutzer die Möglichkeit bieten, Beiträge anderer Benutzer zu kommentieren.
		\item\textbf{\lbrack F160\rbrack} Das System soll dem Benutzer die Möglichkeit bieten, Beiträge andere Benutzer zu bewerten.
		\item\textbf{\lbrack F170\rbrack} Das System soll dem Benutzer die Möglichkeit bieten, durch Aktivität in der Kommunikationsplattform Erolgspunkte zu sammeln.
		\item\textbf{\lbrack F180\rbrack} Das System muss dem Benutzer die Möglichkeit bieten, verschiedene Benutzeroberflächen und Funktionen für verschiedene Benutzergruppen zu benutzen.
		\item\textbf{\lbrack F190\rbrack} Das System soll den Benutzer die Möglichkeit bieten, seine Tagebuch-Einträge in Form einer Tabelle als PDF-Datei zu exportieren und an einer beliebigen E-Mail-Adresse zu senden.
		\item\textbf{\lbrack F200\rbrack} Falls ein zweiter Benutzer um Erlaubnis des Einblicks in die Blutzuckerwerte eines Benutzers gefragt hat, muss das System dem Benutzer die Möglichkeit bieten, die Erlaubnis zu erteilen oder abzulehnen.
	\end{itemize}
	\paragraph{Für Typ-1-Diabetiker}\mbox{}
	\begin{itemize}
		\item\textbf{\lbrack F210\rbrack} Falls der Benutzer eine Mahlzeit als Ereignis hinzufügt, muss das System dem Benutzer die Möglichkeit bieten, die BE's und Insulineinheiten anhand der Kohlenhydrate der Mahlzeit und der Benutzerdaten zu berechnen.
		\item\textbf{\lbrack F220\rbrack} Das System muss dem Benutzer die Möglichkeit bieten, individuellen Insulin- und Korrekturfaktoren anzulegen.
		\item\textbf{\lbrack F230\rbrack} Das System muss dem Benutzer die Möglichkeit bieten, feste Insulinzunahmen anzulegen.
		\item\textbf{\lbrack F240\rbrack} Das System muss dem Benutzer die Möglichkeit bieten, feste Insulinzunahmen zu bearbeiten.
		\item\textbf{\lbrack F250\rbrack} Das System muss dem Benutzer die Möglichkeit bieten, feste Insulinzunahmen zu entfernen.
		\item\textbf{\lbrack F260\rbrack} Das System muss dem Benutzer die Möglichkeit bieten, feste Insulinzunahmen einzusehen.
	\end{itemize}
	\paragraph{Für Typ-2-Diabetiker}\mbox{}
	\begin{itemize}
		\item\textbf{\lbrack F270\rbrack} Das System muss dem Benutzer die Möglichkeit bieten, die Art der Behandlung festzulegen.
		\item\textbf{\lbrack F280\rbrack} Das System soll dem Benutzer die Möglichkeit bieten, sein Zielkörpergewicht anzugeben.
	\end{itemize}
	\paragraph{Für Personen im Umfeld (Eltern, Lebenspartner-/in, ...)}\mbox{}
	\begin{itemize}
		\item\textbf{\lbrack F10\rbrack} Das System muss dem Benutzer die Möglichkeit bieten, ein individuelles Benutzerkonto anzulegen.
		\item\textbf{\lbrack F20\rbrack}  Das System muss dem Benutzer die Möglichkeit bieten, die individuellen Daten seines Benutzerkontos zu bearbeiten.
		\item\textbf{\lbrack F30\rbrack} Das System muss dem Benutzer die Möglichkeit bieten, sein angelegtes Benutzerkonto wieder zu löschen.
		\item\textbf{\lbrack F40\rbrack} Das System muss dem Benutzer die Möglichkeit bieten, unterschiedliche Ereignisse (Blutzuckerwert, Mahlzeit und sportliche Aktivität) manuell in das Tagebuch einzutragen.
		\item\textbf{\lbrack F290\rbrack} Das System muss dem Benutzer die Möglichkeit bieten, verschiedene Benutzeroberflächen und Funktionen für verschiedene Benutzergruppen zu benutzen.
		\item\textbf{\lbrack F300\rbrack} Das System muss dem Benutzer die Möglichkeit bieten, um Erlaubnis des Einblicks in die Blutzuckerwerte eines zweiten Benutzers zu fragen.
		\item\textbf{\lbrack F310\rbrack} Falls der Benutzer die Erlaubnis eines zweiten Benutzers erhalten hat, muss das System dem Benutzer die Möglichkeit bieten, Einblick in Blutzuckerwerte des zweiten Benutzers zu haben.
	\end{itemize}
\subsubsection{Non-funktionale Anforderungen}
\label{section:nfAnforderungen}
\paragraph{Qualitätsanforderungen}
\begin{itemize}
	\item\textbf{\lbrack Q10\rbrack} Das System soll dem Benutzer eine Aufgabenerfüllung innerhalb der Genauigkeits- und Vollständigkeitsgrenzen bieten. (Effektivität)
	\item\textbf{\lbrack Q20\rbrack} Das System soll dem Benutzer eine Aufgabenerfüllung in Bezug auf den Benutzeraufwand bieten. (Effizienz)
	\item\textbf{\lbrack Q30\rbrack} Das System soll dem Benutzer eine von Beeinträchtigungen freie Nutzung und mit einer positiven Einstellung gegenüber dieser bieten. (Zufriedenstellend)
	\item\textbf{\lbrack Q40\rbrack} Das System soll dem Benutzer eine effektive, effiziente und zufriedenstellende Aufgabenerfüllung bieten. (Gebrauchstauglichkeit)
	\item\textbf{\lbrack Q50\rbrack} Das System soll zu 99,9\% erreichbar sein und eine gewisse Ausfallsicherheit garantieren.
	\item\textbf{\lbrack Q60\rbrack} Das System soll über eine strukturierte Benutzeroberfläche mit intuitiver Benutzerführung verfügen.
	\item\textbf{\lbrack Q70\rbrack} Das System muss dem Benutzer fehlerfreie Ergebnisse und Informationen bieten.
\end{itemize}
\paragraph{Organisationale Anforderungen}
\begin{itemize}
	\item\textbf{\lbrack O10\rbrack} Das System soll sensible Daten sicher und unerreichbar für Dritte speichern.
	\item\textbf{\lbrack O20\rbrack} Das System soll einen verlustfreien Datentransport zwischen den verschiedenen Systemkomponenten gewehrleisten.  
	\item\textbf{\lbrack O30\rbrack} Das System soll einen geringen Akkuverbrauch aufweisen.
	\item\textbf{\lbrack O40\rbrack} Das System muss jeder Zeit Kontakt zum Benutzer aufnehmen können.
	\item\textbf{\lbrack O50\rbrack} Das System muss dem Benutzer einen schnellen Zugriff auf den aktuelle Daten bieten.  
\end{itemize}
	Diese Anforderungen werden in den folgenden Aufgaben direkt zum Treffen von Entscheidungen für das User Interface Design verwendet.