\section{Themenfeld/-recherche}
	Bereits im Rahmen des Praxisprojektes im Sommersemester 2019 wurde sich mit der Themenfeldrecherche befasst. Diese ist im Anhang (s. Anhang: A \nameref{section:Themenfeld} ab Seite \pageref{section:Themenfeld}) einzusehen und beinhaltet Informationen über Diabetes mellitus, seine verschiedenen Arten und über die analoge Tagebuch-Dokumentation. Im Zuge dieser Arbeit gilt es das Themenfeld um die im Anwendungsbereich relevanten Themen, Ernährung, Sport und Kommunikation, zu erweitert.
	\subsection{Ernährung}
		Neben der Erhaltung der Lebensqualität ist ein möglichst langes und gesundes Leben ein langfristiges Ziel bei der Behandlung von Diabetes. Eine ausgewogene Ernährung ist hierbei einer der wesentlichen Bausteine. \newline
		Bei einer Studie von DiRECT (Diabetes Remission Clinical Trial), an der 289 Typ-2- Diabetiker teilnahmen, wurden die Teilnehmer in zwei Gruppen geteilt; Eine Kontroll- und eine Interventionsgruppe. Ziel dieser Studie war es, mit den Teilnehmern der Interventionsgruppe eine Veränderung im Ernährungsstil durchzuführen, während die Kontrollgruppe unter Beobachtung von Arztpraxen blieb. Beide Gruppen verzichteten im Zeitraum vom 25.07.2014 bis zum 05.08.2017 auf Insulin, Antidiabetika, Blutdrucksenkende Medikamente und Sport. Der Interventionsgruppe wurden  regelmäßig Einweisungen in die Ernährung gegeben und Energieversorgungsgrenzen festgelegt. Nach 12 Monaten erzielten 36 Interventionsgruppen Teilnehmer (24\%) und kein einziger aus der Kontrollgruppe einen Gewichtverlust von mindestens 15 kg. Der HbA1c-Wert von 86\% aus der Interventionsgruppe und 4\% aus der Kontrollgruppe lag unter 6,5\%. Die Ernährungsumstellung in der Interventionsgruppe verbesserte die Lebensqualität der Teilnehmer und stabilisierte deren Blutdruck, sodass 48\% der Teilnehmer dieser Gruppe auch über die 12 Monaten hinaus keine blutdrucksenkenden Medikamente mehr einnehmen mussten. Diese Studie hat gezeigt, dass eine Gewichtsabnahme  einen gesunden HbA1c-Wert und eine Rückbildung des Diabetes bei Typ-2-Diabetikern erzielt. \\
		Auch bei Menschen ohne Diabetes mellitus bietet eine ausgewogene Ernährung die Möglichkeit, ihre Gesundheit zu stärken und ihr Wohlbefinden zu verbessern.\newline	
		Der Verzehr von Nahrungsmitteln und deren Nährstoffen liefert Energie für den menschlichen Körper und es können Substanzen wie Muskelzellen und Wirkstoffe wie Hormone produziert werden. .\cite{ND}\\
		Ein individueller Ernährungsplan kann bei einer ausgewogenen Ernährung helfen. Es ist Wichtig, dass eine gesunde und bemessene Ernährung nicht nur phasenweise, sondern dauerhaft im Leben geführt wird. Bei der Erstellung eines Ernährungsplans sind der von Körpergröße, Gewicht, Geschlecht, Alter und körperlicher Aktivität abhängige  Energiebedarf, , die persönlichen Ernährungsgewohnheiten und die Therapieform  von großer Bedeutung. Zur Berechnung des Energiebedarfs werden das Körpergewicht, der tägliche Energiebedarf und der Energiegehalt der zugeführten Nährstoffe benötigt.\cite{SG} \newline
		Nach der Harris-Benedict-Formel ergibt sich der Energiebedarf pro Tag aus dem Produkt des Grundumsatzes und des Leistungsumsatzes. Die Formel für den Kaloriengrundumsatz ist abhängig vom Geschlecht, Körpergröße, Körpergewicht und Alter. Der Leistungsumsatz lässt sich von der Intensivität der täglichen Aktivitäten ableiten. So ergeben sich für den Kaloriengrundumsatz folgende Formeln: \newline
		\\
		\centerline{Mann: $Grundumsatz [kcal/ 24h] {=} 66,47 + (13,7 \cdot x) + (5 \cdot y) - (6,8 \cdot z)$;}\\
		\centerline{Frau: $Grundumsatz [kcal/ 24h] {=} 655,1 + (9,6 \cdot x) + (1,8 \cdot y) - (4,7 \cdot z)$;}\\
		\noindent\hspace*{16mm}{x = Körpergewicht [kg];}\newline
		\noindent\hspace*{16mm}{y = Körpergröße [cm];}\newline
		\noindent\hspace*{16mm}{z = Alter [y], \cite{SG}.}\newline
		\\
	Wie  bereits oben erwähnt, wird der Grundumsatz mit dem Leistungsumsatz multipliziert. Hierzu werden die PAL Werte (Physical Activity Level) aus der Tabelle \ref{tab:PAL-Werte}: \nameref{tab:PAL-Werte} verwendet.\cite{SG}
	\begin{table}[H]
		\setlength{\tabcolsep}{12pt}
		\centering
		\begin{tabular}{ll}
			\toprule
			\textbf{PAL-Wert} & \textbf{Aktivität}\\
			\midrule
			0,95 & Schlafen\\
			1,2 & Sitzen/Liegen\\
			1,4-1,5 & kaum körperliche Aktivität\\
			1,6-1,7 & wenig körperliche Aktivität\\
			1,8-1,9 & Stehen/Gehen\\
			2.0-2.4 & körperlich anstrengende Aktivität\\
			\bottomrule
		\end{tabular}
		\captionsetup{justification=centering}
		\caption{PAL-Werte}
		\label{tab:PAL-Werte}
	\end{table}
	\setlength{\parindent}{0pt}Ein Mann mit einem Körpergewicht von 80 kg und einer Körpergröße von 180 cm im Alter von 25 Jahren besitzt einen Grundumsatz von \cite{SG}:\newline
	\\
		\centerline{$66,47 + (13,7 \cdot 80) + (5 \cdot 180) - (6,8 \cdot 2) {=} 1892,47 kcal/24 h $}\newline
	\\
	Geht man nun davon aus, dass er am Tag 8 Stunden schläft, 8 Stunden nur sitzt, 6 Stunden hauptsächlich steht und 2 Stunden anstrengenden Sport betreibt, wird der Leistungsumsatz folgendermaßen berechnet \cite{SG}:\newline
		\\
		\centerline{$\frac{8}{24} \cdot 0,95 + \frac {8}{24} \cdot 1,2 + \frac{6}{24} \cdot 1,9 + \frac {2}{24} \cdot 2,4 {=} 1,39$}\newline
		\\
	Multipliziert man nun den Grundumsatz mit dem Leistungsumsatz, erhält man ca. 2632 kcal/24 h. Mit der Zunahme von 2632 kcal/24 h würde der Herr im Beispiel sein Gewicht halten. Bei Übergewicht müssen täglich 1000 kcal eingespart werden, um wöchentlich 1 kg Körpergewicht zu verlieren.\newline
	Das Normalgewicht kann über den Body-Mass-Index (BMI) berechnet werden. Der BMI lässt sich durch Körpergröße und –gewicht berechnen. Dazu teilt man das Körpergewicht kg durch das Quadrat der Körpergröße m.\newline
	Bei einem BMI von unter 18,5 spricht man von Untergewicht, bei einem zwischen 18,5 und 24,9 ist man normalgewichtig und ab einem BMI von 25 liegt ein Übergewicht vor. Die Zunahme des BMI ist mit zunehmendem Alter normal.\newline
	Die Verteilung und Zusammensetzung der Mahlzeiten pro Tag spielen in der Ernährung eine große Rolle. Empfohlen werden 5-6 kleinere Mahlzeiten pro Tag. So kann der Heißhunger vermieden werden und die Blutzuckerwerte, besonders im Falle einer Diabeteserkrankung, besser kontrolliert werden. Neben dem Frühstück, Mittagessen und Abendessen sollte es drei weitere Zwischenmahlzeiten geben. Hierbei ist eine ausgewogene Ernährung notwendig. Ziel des Ernährungsplans ist es, neben der Ausgewogenheit das Richtige zur richtigen Zeit in der richtigen Menge zu essen. Die Hauptbausteine einer ausgewogenen Ernährung sind die Nährstoffe Kohlenhydrate, Eiweiß und Fett. Für eine erfolgreiche Energieproduktion im Körper, sollte die tägliche Nahrungsaufnahme ein Nährstoffverhältnis von 12-15\% Eiweiß, 25-30\% Fett und 55-60\% Kohlenhydraten aufweisen.\cite{SG} 
		
	\subsubsection{Kohlenhydrate}
		Kohlenhydrate sind energiegewinnende Nährstoffe und werden vom Körper dauerhaft in der Verdauung, Herztätigkeit, Atmung und Bewegung benötigt.\cite{SG} Im menschlichen Körper werden Kohlenhydrate in Energie umgewandelt. Sie existieren in verschiedenen Formen: Einfachzucker liegt als Glukose, (Traubenzucker), als Fruktose (Fruchtzucker) und als Galaktose (Schleimzucker) vor. Zweifach- und Vielfachzucker sind aus mehreren Einfachzuckern zusammengesetzt. Zweifachzucker bestehen aus zwei Einfachzuckern und existieren in Form von Saccharose (Haushaltszucker), Maltose (Malzzucker) und Laktose (Milchzucker).\cite{ND} Der Haushaltzucker oder auch Saccharose besteht beispielsweise aus Glukose und Fruktose, während sich der Milchzucker aus Glukose und Galaktose zusammensetzt.\cite{SG} Einfach- und Zweifachzucker sind schnellwirkende Zuckerarten und sorgen im Körper für eine schnelle aber auch kurz anhaltende Energiezufuhr, da diese für den Körper schnell umzuwandeln sind. Vielfachzucker hat einen komplexeren Aufbau und sein Energieschub hält deutlich länger an. Lebensmittel mit komplexem Zucker enthalten viele Stärke- und Ballaststoffe, welche eine stabilisierende Wirkung auf den Blutzuckerspiegel haben. Durch den Abbau des Vielfachzuckers zu Einfachzucker in der Verdauung, kommt es zu dieser langen Wirkung des Vielfachzuckers.\cite{ND} Kohlenhydrate können nur in der Form des Einfachzuckers durch die Darmschleimhaut ins Blut gelangen. Erst dann steigt auch der Blutzucker an. Aus dem Blut wird der Zucker dann vom Insulin in die Zellen transportieren und somit die Energie aus den Kohlenhydraten gewonnen. Der Blutzuckerspiegel fällt folglich.\cite{SG} \newline
		Viele Lebensmittel, die Vielfachzucker enthalten, enthalten auch Vitamine und noch weiter Nähstoffe. Einfachzucker dagegen nennt man auch „leere Energielieferanten“, da sie keine weiteren Nähstoffe mit sich bringen. Für einen Erwachsenen wird eine Kohlenhydratzufuhr von 55-60\% der gesamten Energiezufuhr pro Tag empfohlen. Davon sollten nicht mehr als 10\% aus Einfachzucker und Zweifachzucker bestehen, wobei der Vielfachzucker den größten Anteil ausmachen sollte. \newline
		Erhält der Körper zu viele Kohlenhydrate, werden diese in Fett umgewandelt und als Reserven in Form von Körperfett gespeichert.\cite{ND} \newline
		Da gerade bei Diabetikern die Aufnahme von Kohlehydrate für einen Anstieg des Blutzuckerspiegels sorgt, spielen die verschiedenen Zuckerarten eine besonders große Rolle. Und da nur gespritztes Insulin den Blutzuckerspiegel wieder senkt, müssen die Kohlenhydrate in einer einheitlichen Rechengröße umgerechnete werden, um die korrekte Insulinmenge ermitteln zu können. Hierzu wurden die Broteinheit (BE) und Kohlenhydrateinheit (KE) eingeführt. Eine BE entspricht 12g Kohlenhydrat und eine KE gleicht 10g Kohlenhydrate. So ergibt sich aus einem Brötchen, das 25 g wiegt und 12 g Kohlenhydrate besitzt, 1 BE bzw. 1,2 KE. Es ist ratsam, sich an die Nährwerttabelle zu halten und die Mengen anhand der Erfahrungen mit Kohlenhydraten abzuschätzen.\newline
		Abhängig von der Struktur der Kohlenhydrate kann die Abbaugeschwindigkeit im Körper, aber auch die Zuckeraufnahme im Blut unterschiedlich sein. Diese Aufnahmegeschwindigkeit beschreibt, wie schnell der Blutzuckerspiegel auch durch die Aufnahme der jeweiligen Kohlenhydrate steigt. Somit existiert eine „Blutzuckerwirksamkeit der Kohlenhydrate“. Der glykämische Index (GI) beschreibt die Wirkung der Kohlenhydrate auf den Blutzucker und ist ein Maß, mit dem diese in Prozenten beschrieben wird. Je höher der GI, desto schneller, je niedriger, desto langsamer steigt der Blutzuckerspiegel. Neben der Wirksamkeit der Kohlenhydrate beschreibt der GI auch die Sättigungsdauer der Kohlhydrate. Kohlenhydrate mit einem niedrigeren GI sättigen länger. \newline
		Eine BE Traubenzucker hat eine GI von 100\%. Weißbrot und Cornflakes besitzen einen GI von >70\% und gehören zu den Lebensmitteln mit einem hohen GI. Der mittlere GI liegt zwischen 55 und 70\%, wie z.B. bei Kartoffeln. Vollkornbrot und Milch besitzen mit <55\% einen niedrigen GI und haben somit eine langsame Wirkung auf das Blut und bringen ein längeres Sättigungsgefühl mit sich.\cite{SG}\\
		Bestehend aus Zucker befinden sich Kohlenhydrate in vielen Lebensmittel. Durch die Komplexität und den gykämischen Index können sich Kohlenhydrate unterscheiden und auch in einem gesunden Körper unterschiedliche Wirkungen erzeugen. Beschäftigt man sich also mit der Frage, ob Diabetiker Zucker essen dürfen, sollte man beachten, dass Früchte wie Äpfel oder Bananen auch Zucker enthalten. Ist dem Diabetiker Zucker zu verbieten, müsste also das Obst ebenfalls verboten werden. Zudem sind auch Leckereien wie Schokolade, Eis oder andere Süßigkeiten bei einem Diabetiker nicht ungesünder als bei einem nicht Erkrankten, solange das nötige Insulin gespritzt wird. Diabetiker müssen nicht Kohlenhydrate sparen, sondern den Kohlenhydrategehalt von Lebensmitteln korrekt bestimmen. Denn das Ziel jedes Diabetikers sind normale Blutzuckerwerte.\cite{SG}		
	\subsubsection{Eiweiß}
		Eiweiße sind Nährstoffe, die zur Energiegewinnung im Körper und als Baustein für Körperzellen dienen. Es existieren 20 verschiedene Aminosäuren, die für den menschlichen Körper wichtig sind und sich vielfältig zum Eiweiß zusammensetzen lassen. Es gibt Aminosäuren, die essentiell sind und der menschliche Körper produziert sie nicht. Er kann sie nur durch die Nahrung aufnehmen. In der Nahrung unterscheidet man zwischen tierischen und pflanzlichen Eiweiße. Tierisches Eiweiß ist dem des Menschen sehr ähnlich und daher wertvoller. Durch die Ernährung sollte der Mensch tierisches und pflanzliches Eiweiß in einem ausgewogenen Verhältnis zu sich nehmen. Eiweiße sind lebensnotwendig und können nicht ersetzt werden.\cite{ND} Sie werden für den Aufbau von Zellen, das Wachstum und für die Blut- und Hormonbildung benötigt. \newline
		Anders als Kohlenhydrat und Fette werden Eiweiße ihm Körper ständig umgewandelt, ab- und aufgebaut, wodurch eine Speicherung der Aminosäuren nicht möglich ist. Eine regelmäßige Aufnahme von Proteinen ist daher von besonderer Wichtigkeit. \newline
		Werden zu viele Eiweiße aufgenommen, wird das überschüssige Eiweiß in Fett umgewandelt und als Reserve gespeichert. Überschüsse können auch in Glykogen, der Speicherform der Kohlenhydrate in der Leber, umgewandelt werden. Dies hat zur Folge, dass Leber und Niere bei dauerhaftem Proteinüberschuss beschädigt werden können und da beide bei hohen Blutzuckerwerten auch schon stark belastet werden, sollte gerade der Diabetiker nicht mehr als die empfohlene Tagesmenge an Eiweiß zu sich nehmen.\cite{SG} Hierbei liegt die Empfehlung bei 0,8 g Eiweiß pro kg Körpergewicht am Tag und 15-20\% der täglichen Gesamtenergieaufnahme. In Deutschland liegt der Schnitt allerdings über diese Empfehlung.\cite{ND}\newline
		Bei einer hohen Eiweißaufnahme durch eine Mahlzeit, steigt die Eiweißkonzentration im Blut für kurze Zeit an, worauf die Bauchspeicheldrüse reagiert und Glukagon freisetzt. Glukagon erhöht die Insulinresistenz des Körpers und in der Leber wird überschüssiges Eiweiß in Glukose umgewandelt. Dadurch steigt der Blutzuckerspiegel.\cite{SG}
	\subsubsection{Fett}
		Fett ist neben Kohlenhydraten und Eiweiß der dritte und stärkste energieerzeugendste Nährstoff und gewinnt doppelt so viel Energie wie die ersten beiden. Fette tragen Aroma- und Geschmacksstoffe und versorgen den Körper mit konzentrierter Energie. Eine übermäßige Fettaufnahme ist auf lange Sicht sowohl für Diabetiker als auch für gesunde Menschen schädlich. Zusätzlich zum Anstieg des Blutdrucks und dem höheren Risiko einer Gefäßverkalkung, führt zu viel Fett zu einem Anstieg der Blutfette. besonders für übergewichtige Menschen und Diabetiker ist ein messbarer und geringer Fettgehalt in der Ernährung wichtig. \cite{SG} Überschüssiges Fett, dessen Energie der Körper nicht verbrennen kann, wird als Körperfett angelegt. Die Gewichtszunahme ist einer der Folgen bei dauerhaftem Überschuss an Fett im menschlichen Körper. Erlangt der Körper dauerhaft zu wenig Fett, baut dieser Körperfett zur Energieverbrennung ab und nimmt ab.\cite{ND} Da Fette hochkonzentrierte Energie produzieren, spart die Reduzierung der Fettzunahme schnell viele Kalorien, ohne die Ernährungsgewohnheiten zu beeinträchtigen.\cite{SG}\newline
		Fette unterscheiden sich in ihrem Gehalt an gesättigten und ungesättigten Fettsäuren. Beides sind Bausteine von Fetten. Gesättigte Fettsäuren kommen in tierischen Fetten vor und ungesättigte Fettsäuren sind in pflanzlichen Fetten enthalten. Letztere kann der menschliche Körper nicht selber produzieren und sind in der Ernährung notwendig. Ein gesundes Verhältnis der verschiedenen Fettsäuren muss eingehalten werden. Dabei gilt bei Erwachsen einen Fettanteil von 25-30\% in der täglichen Energiezufuhr einzuhalten. Das entspricht 60 bis 80 g Fett pro Tag. Da die pflanzlichen Fette gesünder sind, sollten sie den größeren Teil der zugeführten Fette ausmachen. Je mehr gesättigte Fettsäuren das Fett enthält, desto dicker ist seine Konsistenz. \newline
		Omega-6- und Omega-3-Fettsäuren sind essentielle pflanzliche Fettsäuren und werden zur Bildung von funktionell notwendigen Fettstrukturen benötigt. Der Bedarf an beiden kann durch eine ausgewogene Aufnahme verschiedener pflanzlichen Fetten gedeckt werden. Omega-6-Fettsäuren bewirken eine Senkung des Cholesterinspiegels im Blut und Omega-3-Fettsäuren verbessern das Fließen des Blutes im Körper. Zudem stärken beide das Immunsystem und helfen gegen Entzündungen im Körper. Bei der Zunahme wird ein Verhältnis von 5-mal so viel Omega-6-Fettsäuren wie Omega-3-Fettsäuren empfohlen. Beide sind in diversen pflanzlichen Ölen, wie Sonnenblumen-, Mais-, oder Rapsöle vorhanden, aber auch Seefische enthalten Omega-3-Fettsäuren.\cite{ND}\newline
		Fettige Nahrung benötigt länger um im Margen verdaut zu werden. Werden also viele Fette und viele Kohlenhydrate aufgenommen, so ist mit einer dauerhaften Kohlenhydratzufuhr zu rechnen. Ein verzögerter Effekt auf den Blutzuckerspiegel ist daher bei einem Diabetiker zu erwarten. \cite{SG}\\
		Abschließend ist festzuhalten, dass die Energiegewinnung durch die Ernährung für den menschlichen Körper, unabhängig ob Diabetiker oder gesunder Mensch, notwendig ist. Ein gesunder Mensch hat genauso auf seine Ernährung zu achten, wie ein Diabetiker, nur muss er seine Nährstoffe nicht zählen, in Brot- und Insulineinheiten umrechnen und Insulin injizieren. Das Hauptziel eines Diabetikers sind Blutzuckerwerte im gesunden Bereich. Die Ernährung ist dabei von großer Bedeutung.

\subsection{Sport}
	Neben der Ernährung ist Bewegung ein wichtiger Bestandteil bei der Therapie und Behandlung von Diabetes. Sportliche Betätigung stabilisiert nicht nur den Blutzuckerspiegel und senkt das Risiko von Folgeerkrankungen, sondern verbessert auch die Lebensqualität und des Wohlbefindens aller.\cite{SG}\\
	Gerade bei der Erkrankung an Typ-2-Diabetes ist eine regelmäßige körperliche Aktivität von großer Bedeutung. Sie ist eine Folge von Bewegungsmangel, schlechter Ernährung und Fettleibigkeit. Vor einigen Jahren war diese Art von Diabetes als „Alters-Diabetes“ bekannt, aber die Zahl der erkrankten Kinder, Jugendlichen und jungen Erwachsenen nimmt stetig zu. Sie ist laut dem IDF (International Diabetes Federation) die häufigste Form des Diabetes und macht 90\% der Prävalenz aus.\cite{IDF}\newline
	Ein Mangel an Aktivität erhöht die Insulinresistenz und regelmäße Bewegung ist bei allen Formen von Diabetes ratsam. Je mehr Bewegung der Körper bekommt, desto mehr Gesundheit, Fitness und Spaß strahlt er aus. Bereits ein Leistungsumsatz von 2000 kcal pro Woche reduziert das Risiko einer Durchblutungsstörung des Herzens um bis zu 60\%. Es muss kein Extremsport sein. Regelmäßige Bewegungen im Alltag, wie zum Beispiel Treppensteigen statt Fahren mit dem Aufzug, würden es leichter machen, die nötige Leistung zu erbringen. \newline
	Im Falle eines Übergewichtes ist deine Gewichtsabnahme nur durch langfristige, gesunde und kontinuierliche Aktivität in Kombination mit einer ausgewogenen Ernährung möglich. Hier gilt auch, auf die Kalorienzufuhr zu achten. Eine tägliche Reduzierung des Energiebedarfes um 1000 kcal ist gefährlich und baut Eiweiß im Körper ab, wodurch sich der Grundumsatz automatisch senkt.\newline 
	Bei regelmäßigem und aus 70\% Ausdauer-, 10\% Kraft- und 20\% Geschicklichkeit bestehendem Training ist eine Gewichtsabnahme von 0,5-0,7 kg pro Woche realistisch. Wichtig ist dabei, einen großen Teil der gesamten Muskulatur zu beanspruchen.\newline
	Auch mit zunehmendem Alter spielt die sportliche Aktivität eine Rolle. Im Alter von 30 Jahren beginnt der Körper, jährlich Muskulatur und Knochenmasse abzubauen.  Dies lässt sich durch regelmäßigen Sport verhinderen bzw. verlangsamen.\newline
	Sport senkt den Blutzuckerspiegel im Körper. Bei einem Diabetiker besteht jeder Zeit die Gefahr einer Hypoglykämie (Unterzuckerung). In einem gesunden Körper verhindert eine rechtzeitige Einstellung der Insulinproduktion dies und sorgt eigenständig für die Stabilisierung des Blutzuckers über 50 mg/dL. \cite{SG}\newline
	Beim Sport können schon wenige Insulineinheiten große Wirkungen auf den Blutzucker haben, da körperliche Aktivität nicht nur die Insulinresistenz senkt, sondern auch die Insulinsensitivität erhöht. Der Blutzuckerspiegel könnte bei einem Diabetiker so stark sinken, dass es zu einer schweren Unterzuckerung kommt. Deswegen ist es notwendig bei geplanten sportlichen Aktivitäten, die Insulinzufuhr zu reduzieren und vorsichtshalber zusätzliche Kohlenhydrate einzunehmen - diese werden auch als „Zusätz-BE‘s“ bezeichnet. Zudem verursacht eine Reduzierung der Insulinzufuhr die Freisetzung von Zuckerreserven (Glykogen) in den Muskeln und die Produktion von Glukose in der Leber. Besonders nach längeren Aktivitäten ist mit einem späteren Abfall des Blutzuckers zu rechnen, da die Insulinempfindlichkeit auch nach dem Sport anhalten kann und die Glykogen Speicher in den Muskelzellen wieder aufgefüllt werden müssen. \cite{RP}\newline
	Bei Menschen, die regelmäßig Sport treiben, sind mehr Zuckerreserven in den Muskeln vorhanden. Ein Diabetiker, der nicht regelmäßig Sport treibt, hat ein höheres Risiko für Hypoglykämie. Ein sportlicher Diabetiker muss beachten, dass bei der Beanspruchung von Muskelgruppen, die ansonsten selten beansprucht wurden, ebenfalls wenige Glykogenspeicher in den Muskeln vorhanden sind. \newline
	Bei ungeplanter sportlichen Aktivität ist die Vermeidung einer Unterzuckerung meist nur durch die Aufnahme von Zusatz-BE’s möglich, da die Reduzierung der Insulinzufuhr meistens zu spät wirkt.\newline
	Neben der Hypoglykämie-Gefahr besteht während des Sportes auch die Gefahr einer HyperglykämieIst der Diabetiker bereits vor der sportlichen Aktivität überzuckert, liegt ein Insulinmangel vor. Dieser Insulinmangel  verhindert eine bessere Glukoseverwertung durch die Muskeln und fördert eine übermäßige Glukoseproduktion in der Leber. Diese Überproduktion hat zur Folge, dass der Blutzuckerspiegel weiter ansteigt. Insulinmangel bedeutet auch, dass keine Energie gewonnen und zu den Muskelzellen transportiert werden kann. Der Körper liefert diese aus seinen Fettreserven. Das Ergebnis ist die Fettverbrennung, wodurch die Ketonkörperbildung steigt. Diese Ketone säuern das Blut an und schädigen Organe wie Nieren und Leber. Kämen noch Sportaktivitäten dazu, würden noch mehr Fett abgebaut und die Ketonkörperbildung im Köper gefördert werden. Folglich kann es zu einer diabetischen Entgleisung (Ketoazidose) kommen, die bis zum Koma führen könnte. Deshalb sollte bei einem hohen Blutzuckerspiegel kein Sport getrieben werden. Auch Hormone wie Adrenalin, die beim Sport vom Körper ausgeschüttet werden können einen weiteren Blutzuckeranstieg verursachen. \newline
	Ist der Blutzuckerspiegel eines Diabetikers vor dem Sport im Zielbereich, so ist zu beachten, dass nur so viele Zusatz-BE’s aufgenommen werden, wie notwendig. Oft ist die Schwere der sportlichen Aktivität nicht vorher zu sehen, weshalb zusätzliche Kohlenhydrate in kleinen Mengen und in kurzen Abständen aufzunehmen sind.  \cite{SG}\\
	Fazit ist, der Sport ist neben der Ernährung im Leben von großer Bedeutung. Besonders Diabetiker müssen noch einige Dinge beachten. So ist die richtige Menge an Zusatz-BE’s zu ermitteln und an eine frühzeitige Insulinreduzierung zu denken. Der Sport wirkt nach und kann noch bis zu 14 Stunden nach der Aktivität Einfluss auf den Blutzuckerspiegel und die Insulinempfindlichkeit haben, denn die Glukosespeicher müssen in der Leber und im Muskel wieder aufgefüllt werden. Bei zu hohem Blutzucker ist vom Sport abzuraten bzw. muss er sofort abgebrochen werden. Bei einer Unterzuckerung sollte bis zur Stabilisierung des Blutzuckers eine kurze Pause eingelegt werden. Das regelmäßige Blutzuckermessen vor, während und nach dem Sport ist essentiell, um Risiken während des Sportes vorbeugen zu können. \cite{SG}
\subsection{Kommunikation/Erfahrungsaustausch}
	Im Leben eines Diabetikers stellen sich häufig Fragen zur Krankheit, die durch den Austausch zwischen Diabetikern geklärt werden können. Oft gibt es auch Fragen, die ein Arzt nicht unbedingt beantworten kann und die Antwort darauf liegt eher in der Erfahrung mit Diabetes\cite{JR}\\
	Soziale Kontakte zwischen Diabetikern, aber auch mit gesunden Menschen sind wichtig. Soziale Bindungen können die Lebensqualität und Gesundheit eines Menschen beeinflussen. Ein soziales Netzwerk stützt die mentale Aufrechterhaltung der Menschen. Und soziale Gruppen von Menschen mit ähnlichen Lebensbedingungen, wie Diabetes, können sich positiv auf das Leben auswirken. \newline
	Für Diabetiker treten im Alltag häufig Probleme menschlicher und nicht unbedingt medizinischer Natur auf. Sie werden in kleinen und jungen Jahren in Kindergärten und Schulen und später bei der Arbeit oft benachteiligt und gar diskriminiert. Und wenn es um die Berufswahl geht, gibt es nicht selten Einschränkungen. Der Austausch von Erfahrungen und Tipps unter Diabetikern ist daher wichtig. Beispielsweise können Diabetikersportgruppen gebildet werden, um den Spaß an Sport aufrechtzuerhalten.  Eine solche Gruppe dient in erster Linie dem Erfahrungsaustausch, der Motivation zum Sport und der medizinischen Unterstützung. Viele Diabetiker haben große Angst vor den Folgeschäden der Erkrankung. Der Austausch mit anderen Diabetikern könnte hier beruhigend wirken. \cite{SG}\newline
	Der Deutsche Diabetiker Bund (DDB) ist die größte deutsche Ansprechstelle für Diabetiker und veranstaltet regelmäßig Events und öffentliche Aktionen um Diabetiker zusammen zu bringen. Der DDB setzt sich als Ziel, die Gesundheit und sozialen Kontakte von Diabetikern zu fördern.\cite{JR}\\
	In meiner ersten Studie in einer vergangenen Entwicklungsphase (s. Anhang: A \nameref{section:Evaluation} ab Seite \pageref{section:Evaluation}) wurde festgestellt, dass 69 der 81 Probanden und damit mehr als 87\% aller Teilnehmer Fragen zum Diabetes hatten, 85\% aller Probanden gaben an, bereits online nach Antworten gesucht zu haben und rund 88\% hatten schon Erfahrungsaustausch mit anderen Diabetikern gewünscht.\\
	Kommunikation ist wichtig und trägt zum Wohlbefinden aller bei. Trotz Exsitenz von Organisationen wie des DDB,  ist die Suche nach Kontakten zwischen Diabetikern groß.  Der Kontakt zu anderen Menschen ist aber auch wichtig, um Diskriminierung zu verhindern. Der Erfahrungsaustausch und der soziale Kontakt zwischen Diabetikern sind eine Säule in der Therapie des Diabetes mellitus.
